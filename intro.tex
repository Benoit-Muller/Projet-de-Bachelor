\chapter*{Introduction}
\addcontentsline{toc}{chapter}{Introduction} \markboth{INTRODUCTION}{}
In the following paragraphs, we detail the composition of this project and develop the ideas of the main results.
\\\\
In the first chapter we introduce linear dynamical systems, these are autonomous differential systems, where the derivative of a solution is linear with respect to the solution itself. We write it as a matrix product. Solutions always exist on the all real space and are unique for a same initialisation.

We describe the space of solutions. It has the same dimension as the image space because the linear dependence of solutions is the same as their linear dependence at any fixed time. We can compute the solution starting at any point by multiply the initial position by a matrix called fundamental matrix. Its value can be computed as an infinite sum. We allow our self complex solutions, seeing that their real and imaginary parts are real solutions too, and we find the general form of the solutions, by treating particular solutions that start on the generalised eigenvectors of the system matrix. They are called eigensolutions and are the product of two factors: an exponential and a polynomial. The eigenvalue appears in the exponential and the polynomial is of coefficients generalised eigenvectors of the same eigenspace. A basis of generalised eigenvectors form a basis of eigensolutions for the solution space.

Then we introduce the stability of a fixed point. We say it is Lyapunov-stable when all solutions that start near enough the fixed point never leave an arbitrary neighbourhood. It is asymptotically stable if in addition to that, all solutions that start near enough the fixed point, converge to it. From the understanding of the solutions, we deduce necessary and sufficient conditions on the matrix that define the equation, to have stability on the origin. All conditions relate to the sign of the real part of its eigenvalues, which appear in the argument of the exponential of the eigensolutions. It is asymptotically stable if and only if all eigenvalues have a negative part, and we call such eigenvalues stable. It is Lyapunov-stable if and only if all eigenvalues have a non-positive real part, and a negative one when the eigenvalue is defective. We give a classification of linear dynamical systems, that describe the kind of stability they have. They take into account the range of stability of the eigenvalues.

We present a direct sum that decompose the space of solutions in three subspaces: the generalised eigenspaces associated to eigenvalues of real parts that are respectively negative, null, and positive. These three spaces, called respectively stable, center and unstable eigenspace have a second description that only take the asymptotic comportment of solutions when t tends to plus or minus infinity. Two system that have a trivial center eigenspace are actually linked by a homeomorphism between their solutions. We show a direct application of our understanding of the linear case, the linearization theorem. It says that if the linearization of the system in a fixed point is asymptotically stable, then the nonlinear system is asymptotically stable too. 
\\ \\
We motivate the study of a particular nonlinear dynamical system on the plane, called \LV equations, by showing how it is linked to population dynamics. The functions that are set in an equation are the size of two population of prey and predators. It is based on the concept of logistic growth and the understanding of the growth rate, which is the derivative divided by the function itself. It represent the mean of change an individual add to the population by unit of time. This rate of growth is linear in the \LV equation. We present a modification on it, that change the affect between the species. The idea is to consider in the equation not just the effective size of population, but a function of it. This is useful to have a better description of the affect each species have on the others. We consider two situations. In the first system, a function does not appears in its own growth rate, and in the second it does. 

We investigate the nature of the solutions, for the first system, the linearization in a particular fixed point gives an elliptic linear system, a system with cyclic solutions. We search then for a constant quantity that is conserved along time, and this gives a implicit equation for the trajectory of the solutions. From this, we conclude that the solutions are indeed cyclic, they orbit around the fixed point. In the second system, the linearization in a particular fixed point gives an stable linear system, so we know that the fixed point is asymptotically stable. Motivated by this result, and inspired by the conserved quantity of the first system, we develop the theory of Lyapunov functions. They are function that represent the energy of a position, and such that all solutions loose energy along time. When such a function exist and has a minimum in a fixed point, we can obtain asymptotic stability without condition on the starting point of the solutions, they all converge to the fixed point. Taking a similar expression of the constant quantity of the first system, we find a Lyapunov function. From this, we conclude that all the solutions turn in a spiral around the fixed point and converge to it, we have global asymptotic stability.
\\ \\
From the study of this problem, and the basic strategic idea of this modelisation, we present adaptations to epidemiology dynamics. We investigate the modelisation of a population that is confronted to a virus. We use the \LV system to describe the population of vulnerable people that have not been infected, the population of infected people, and the population of people that have recovered after infection. We see that this model has no isolated fixed point so fixed points are not asymptotically stable. We derive again constant quantities along time, that give implicit formulas for the trajectories of solutions and the limits of them. If the initial number of vulnerable people is bigger than a certain threshold, the number of infected people increase to a maximum located in this threshold, and eventually tends to zero. If the initial number of vulnerable people is smaller than this threshold, the number of infected people directly tends to zero. We give some description of the asymptotic comportment and the reminded size of groups.
In a second modelisation, we introduce the vaccine to the population and separate each group in two, regarding if they are vaccinated or not. The analysis of the problem can be done with the same tools as the precedent, with a bit more work. We obtain a pair of equations that gives some explicit relations between the functions. In particular, we have a bijective relation between the vulnerable groups and show the affect of the vaccine.
\thispagestyle{plain}
