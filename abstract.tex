\begin{abstract}
  The goal of this project is to study the stability of some linear and nonlinear autonomous ordinary differential systems including the "predator–prey" based Lotka-Volterra models, and to further find applications to epidemiology such as the Covid-19 evolution.
  
   The first chapter is devoted to the investigation of the stability of linear autonomous ODE systems, where a complete classification is given thanks to spectrum information. We refine these results to give a stability decomposition of the solution space, and show that in general, some stability properties are preserved to related nonlinear systems thanks to linearization arguments. 
   
   Then, we turn in the second chapter to a more specific form of ODE systems, say Lotka-Volterra models. Instead of considering classical Lotka-Volterra systems, we look at a more general form. Because at some special critical points the linearization stability theory is not always applicable, we introduced the Lyapunov approach, which can even lead to global stability results. 
   
   Finally, the third chapter is an application of the preceding chapters, where we try to model Covid-19 systems concerning patients and vaccinations, and to predict the evolution of the pandemic via stability investigation.
\end{abstract}

%We introduce linear dynamical systems and investigate the general form of the solutions, as well as the space they form. We introduce the stability of a fixed point, which describe the boundedness or the asymptotic comportment of the distance between a fixed point and a solution starting near it. From the understanding of the solutions, we deduce necessary and sufficient conditions on the matrix that define the equation, to have stability. All conditions relate to the sign of the real part of its eigenvalues. We present a classification of linear dynamical systems, that describe the kind of stability they have. We show some links between linear and nonlinear dynamical systems, such as the linearization of a system. We motivate the study of a particular nonlinear dynamical system on the plane, by showing how it is linked to population dynamics, and present a modification on it that change the affect between the species. We investigate the nature of the solutions and the stability of a particular fixed point. From the study of this problem, and the basic strategic idea of this modelisation, we present adaptations to epidemiology dynamics. We derive implicit formulas for the trajectories of their solutions and some descriptions of the asymptotic comportment.