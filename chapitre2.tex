\chapter{Lokta-Voltera Equations and Application}

The classic model of Lokta-Volterra equations is 
\[\dot{x}=x(\a-\b y) \quad  \dot{y} = y(-\g + \d x)\]
where $x$ and $y$ reprensent the size of the population of preys, and predators respectively. The parameters $\a,\b,\g,\d$ are positive scalars. We present a motivation behind this modelisation by showing how we can arrive to this choice of equation:

\section{Motivation} \label{sec:motiv}
We remind that for a function $x$ of one variable $t$, 
\[x'(t) = \frac{\dd}{\dd t}x(t) = \dotx(t) \lim_{h\to0}\frac{x(t+h)-x(t)}{h}\] 
is the derivative when the limit exist. Qualitatively it indicate the amount of change of the function with respect to the variable. This amount is absolute and doesn't depend of the size of x. In consequence, when $x$ is non null which is an assumption we will always make for initial value, $\frac{\dot{x}}{x}$ quantify the relative rate of change of the function, given by the mean number of descendant of a individual. For example if a specie reproduce itself always with the same speed no matter the population or the environment, we can say that their grow rate is a constant c:
\[ \dot{x} = cx, \]
If the population is positive, 
\[ \frac{\dot{x}}{x} = \frac{d}{dt}(\log{x}) = c \]
and then by integrating,
\[ \log{x(t)} = ct + \log{x(0)}, \]
giving us $x(t) = x(0)e^{ct}$, the exponential growth.

Obviously this is doesn't encapsulate the reality as the function grow very fast forever. The growth rate must decrease as the population increase. This come from multiple complex reasons such as envirenment capacities in food, space etc\dots For now we suppose it is from the simplest form, a linear decrease of this growth rate: 
\[ \dot{x} = x(\a -\b x) \]
That give us the logistic growth, where $\a$ represent the initial growth rate, and $\b$ change how fast the growth rate slow down as the population size increase. Here we have one non trivial equilibrium when $\dot{x(t)}=0$ i.e. when $x(t)=\a/\b=x_*$. If not, we remark that $\dotx > 0$ when $ x < x_*$, and  $\dotx < 0$ when $ x < x_*$, meaning that $x_*$ seems stable. indeed we do the computations :
\begin{IEEEeqnarray*}{rCl}
   \frac1\a = \frac{\dotx}{x(1-x/x_*)}
   = \frac{\dotx}{x} + \frac{\dotx}{(x_* - x)}
   = (\log|x| + \log|x_* - x|)'.
\end{IEEEeqnarray*}
By integrating from 0 to t we get
\[ t/\a  = \log(x(t)) - \log|x_* -x(t)| -( \log(x_0) - \log|x_* -x_0|)
= \log \frac{x(t)(x_* -x_0)}{x_0(x_*-x(t))} \]
and rearranging terms
\[x(t) = \frac{x_0x_*e^{\a t}}{x_* + x_0(e^{\a t}-1)} \]
which is well defined for $t\in\R$ if $0<x_0<x_*$, and for $t\in[1/\a \log(1 - x_*/x_0),\infty]$ if $0<x_*<x_0$. We see that in both cases $x(t)\to x_*$ as $t\to\infty$.

In conclusion for this logistic growth, the population will always stabilise in the direction of a unique non trivial equilibrium.


Now we want to introduce a second specie, the predator which alter the growth of the prey population. That mean that the growth rate will decrease in function of the prey, let's say linearly :
\[ \dotx = x(\a -\b y).\]
In the opposite, the growth rate of the predator increase together with the population. But it decrease without them :
\[ \doty = x(-\g +\d x).\]
This is the Lokta-Volterra equations for a prey-predator system.

\section{*Le papier sur les phages marins*}
The idea is that we want to be able to modify the way the size of the species affect the rate. Traditional Lokta-Volterra equation consider the rate of growth as linear in the species. We use then functions $F$ and $G$ that are strictly monotonic,hence bijective, and increase from $0$ to $\infty$ to replace the linearity:
\begin{equation} \label{eq:LV}
    \begin{cases}
    \dotB = F(B)(\a -\b G(\P)) \\
    \dotP = G(\P)(-\g +\d F(B))
    \end{cases}
\end{equation}
Here again, $\a,\b,\g,\d$ are positive scalars.
\\ \\
We search for non trivial equilibrium where $\dotB=\dotP=0$ :
\[ B_* = F^{-1}(\g/\d) \quad \P_* = G^{-1}(\a/\b)\]
This let us rewrite \prettyref{eq:LV} using this notation :
\begin{equation} \label{eq:LV*}
    \begin{cases}
    \dotB = \b F(B)(G(\P_*) - G(\P)) \\
    \dotP = \d G(\P)(-F(B_*) + F(B))
    \end{cases}
\end{equation}
Now we can study the sign of the derivatives $\dotB$ and $\dotP$ and draw a phase plane. The positive values $B_*$ and $\P_*$ divide the positive plane $\R_+^2$ in four regions.
\com{phase plane}
Trajectories seem to turn around the center of equilibrium and we search for a first integral by taking the cross product of \prettyref{eq:LV*} and dividing by $F(B)G(\P)$ :
\begin{IEEEeqnarray*}{rCl} 
    0 &=& \big(\dotB \d G(\P)(-F(B_*) + F(B))
        - \dotP \b F(B)(G(\P_*) - G(\P)) \big) \frac{1}{F(B)G(\P)} 
        \IEEEyesnumber \label{eq:1integral} \\
    &=& \d(-\frac{\dotB}{F(B)}F(B_*) + \dotB)
    -  \b(\frac{\dotP}{G(\P)}G(\P_*) - \dotP) \\
    &=& \Big(\d (-P(B)F(B_*) + B) -\b(Q(\P) G(\P_*))-\P)\Big)'
\end{IEEEeqnarray*}
where $P$ and $Q$ are primitives of $1/F$ and $1/G$, and exist because $F$ and $G$ are continuous. This give us a conserved quantity 
\begin{IEEEeqnarray*}{rCl}
V(B,\P) &=& \d \big(-P(B)F(B_*) + B\big) 
-\b\big(Q(\P) G(\P_*)-\P\big) \\
&=& V(B(0),\P(0)) \\
&=:& V_0
\end{IEEEeqnarray*}
Now we know then that $(B,\P) \in V^{-1}(\{V_0\})$, a closed set as V is continuous.
\\
\com{arguments for the fact that this is a cyclic orbit under asumptions on F and G}
\\
\com{plot $\to$ \url{https://www.desmos.com/calculator/nqxtcrysok}} .
Now that we know that it's cyclic with a period $\tau$, we derive a modified Voltera principle by integrating these qualities obtained from \prettyref{eq:LV*} :
\[ \dotB / F(B) = \b (G(\P_*) - G(\P)). \]
the left side gives
\begin{IEEEeqnarray*}{rCl}
    \int_0^\tau \frac{\dotB}{F(B)} 
     &=& \int_0^\tau (P(B))'
     = P(B(\tau))-P(B(0)) =  P(B(0))-P(B(0)) =0,
\end{IEEEeqnarray*}
and then the rigth side
  \[ 0= \int_0^\tau \b (G(\P_*) - G(\P))
    = \tau\b G(\P_*) - \int_0^\tau \b  G(\P) \]
which implies that
 \[  G(\P_*) = \frac1\tau\int_0^\tau G(\P). \]
 Note that if $G=$Id, this says that the mean of the population $\P$ during time is $F(B_*)$. By the same argument with the second equation, we obtain that similarly
  \[  F(B_*) = \frac1\tau\int_0^\tau F(B). \]
  \\ \\
  In \prettyref{sec:motiv}, we explained how the own size of a population can affect its growth. We presented the logistic growth consequence of this equation :
  \[\dotx = x(\a - \b x)\]
  We add then a term in the rates of \prettyref{eq:LV} to modelise the fact that the growth rate of a population decrease with the size of population due to environment capabilities or competition. Again we use $F$ and $G$ to quantify their importance :
  \begin{equation} \label{eq:LV2}
    \begin{cases}
    \dotB &= F(B)(\a -\b G(\P) - \m F(B)) \\
    \dotP &= G(\P)(-\g +\d F(B) - \n G(\P))
    \end{cases}
\end{equation}
with new positive scalars $\m$ and $\n$. We search for non trivial equilibrium where $\dotB_{**}=\dotP_{**}=0$ and obtain a linear system in $F(B_{**})$ and $G(\P_{**})$ :
\begin{equation*}
    \begin{cases}
    -\a = - \m F(B_{**}) -\b G(\P_{**}) \\
    \g = \d F(B_{**}) - \n G(\P_{**}) 
    \end{cases}
\end{equation*}
which give
\begin{IEEEeqnarray*}{rCl}
    \m\g-\d\a &=& \m\d F(B_{**}) - \m\n G(\P_{**}) - \d\m F(B_{**}) -\d\b G(\P_{**}) \\  &=& (-\m\n-\d\b) G(\P_{**}) 
\end{IEEEeqnarray*}
and then as $\m\n+\d\b > 0$ and supposing $\a\d>\g\m$
\[ G(\P_{**}) = \frac{\a\d-\g\m}{\b\d+\n\m} ,\quad
\P_{**}=P^{-1}\Big(\frac{\a\d-\g\m}{\b\d+\n\m}\Big) \]
Similarly,
\[ F(B_{**}) = \frac{\b\g+\n\a}{\b\d+\n\m}, \quad
B_{**} = F^{-1} \Big(\frac{\b\g+\n\a}{\b\d+\n\m}\Big) \]
Here we cannot derive a first integral like we did in \prettyref{eq:1integral} by separating variables $B$ and $\P$. Instead, we want to test the stability of $(\Bstar,\Pstar)$. For this we develop the theory of stability of Lianupov.
\\ \\
Consider a differential equation $\bdotx=\mathbf{F}(\mathbf{x})$, such that there exist a unique solution for each initial point and for all $t\geq0$. Such solutions are denoted by the flow $\phi$, such that $t\mapsto \phi(\mathbf{x}_0,t)$ is the solution initialised at $\mathbf{x}_0$. 

\begin{definition}
 A fixed point $\mathbf{x}_*$ of  is said \emph{Lyanupov stable} or \emph{L-stable} if 
 for all $\epsilon>0$, there exists a $\delta>0$ s.t. for all $\mathbf{x_0}$ and for all $t>0$, $\|\mathbf{x_0} - \mathbf{x}_*\| < \delta$ implies $\|\phi(\mathbf{x_0},t) - \mathbf{x}_*\| < \epsilon$.
\end{definition}

\begin{definition}
A fixed point $\mathbf{x}_*$ of  is said \emph{attracting} if there exists a $\delta>0$ s.t. for all $\mathbf{x_0}$, $\|\mathbf{x_0} - \mathbf{x}_*\| < \delta$ implies that $\phi(\mathbf{x_0},t) \to \mathbf{x}_*$ as $t \to\infty$.
\end{definition}

\begin{definition}
    A fixed point which is L-stable and attracting is said \emph{asymptoticaly stable}
\end{definition}

\begin{remarque}
Note that these two notions are different. For an example of non-attracting point which is L-stable, we can take simply $F(x)=0$ or for a non trivial case, we can take $\dotx = -y$, $\dot{y} = x$ who describe the circle trajectories $x(t)=cos(t)$, $y(t)=sin(t)$.

In the other way there exist non L-stable points which are attracting. Such a point is the limit of all near trajectories but they always go far before converging, like a detour. For this we place our-self in polar coordinates. We want the trajectories to follow the circle and finish in $(1,0)$ for this we make the $\theta$ always go and stop to $2\pi$, and $r$ go and stop to 1. For this we can write $(\dot{r},\dot{\theta})=(1-r,2\pi-\theta) = G(r,\theta)$. But if we want $(G_1\cos(G_2),G_1\sin(G_2))$ to be continuous on $\R_+\times 0$, we should write $\dot{r} = r(1-r)$, $\dot{\theta} = \theta(2\pi-\theta)$. And to obtain the continuity of the derivative, in purpose to have a flow, we should write $\dot{r} = r(1-r^2)$, $\dot{\theta} = \theta(2\pi-\theta^2)$. This gives us what we need but to be able to explicitly change the coordinates into cartesian, we prefer $\dot{y} = \sin(\theta/2)^2$.
\end{remarque}

Now we can introduce a tool that will be useful to understand the limit comportment of the trajectories and will a tool to proove asymptotic stability.

\begin{definition}
    Assume $\mathbf{x}_*$ is a fixed point of a equation $\bdotx = \mathbf{F}(\mathbf{x})$, and let a function $L:U\to\R$ defined in a neighbourhood of $\mathbf{x}_*$.
    \\
    The function $L$ is called a \emph{weak Lyanupov function} if \[L(\mathbf{x})>L(\mathbf{x}_*)\] 
    and 
    \[ \dot{L}(\xx):=\frac{\dd}{\dd t}L(\phi(\mathbf{x},0)) \leq 0 \]
    for all $\mathbf{x}$ in $U$.
    \\
    The function $L$ is called a \emph{Lyanupov function} (or a \emph{strict Lyanupov function}) provided that we have the strict inequality $\dot{L}(\xx) < 0$ when $\xx\neq\xx_*$ and equality $\dot{L}(\xx_*) = 0$ (In this case we say that $\dot{L}$ is semipositive definite in $\xx_*$, and $L(\mathbf{x})>L(\mathbf{x}_*)$ guarantee anyway).
\end{definition}

\begin{theoreme} \label{th:Lianupov}
Let $\mathbf{x}_*$ a fixed point of the differential equation $\bdotx = \mathbf{F}(\mathbf{x})$ and $L$ a weak Lyanupov function on $U\ni\mathbf{x}_*$. Then $\mathbf{x}_*$ is L-stable. If $L$ is a strict lyanupov function, $\mathbf{x}_*$ is attracting too and thus $L$ is asymptotically stable.
\end{theoreme}
\begin{proof}
L-stable: Let's suppose first that $L$ is a weak Lyanupov function. Since $\dot{L}(\xx) \leq 0$, $L(\xx)$ is decreasing. Let be $\epsilon>0$. Up to taking it smaller, we suppose $B(\xx_*,\epsilon) \subset U$. But $L$ is continuous and $\partial B(\xx_*,\epsilon)$ is compact, we can then define
\[m=\min_{\partial B(\xx_*,\d)}{L} \; \geq \; L(\xx_*),\]
and since $L$ is continuous, there exists a $\epsilon>\d>0$ such that $L(\xx)<m$ when $\|\xx-\xx_*\|<\d$. Now, for all $\xx\in B(\xx_*,\d) \subset B(\xx_*,\epsilon)$, $L(\xx)<m$ mean that $\phi(x,t)$ cannot go out of $B(\xx_*,\epsilon)$, otherwise it would cross $\partial B(\xx_*,\epsilon)$ in a certain $t>0$, and  we would have the contradiction $m<L(\xx)=L(\phi(\xx,0)) \geq L(\phi(\xx,t)) \geq m$. Whave the L-stability.

Attracting: Since $L(\xx)$ is decreasing and bounded below $L(\xx_*)$, $L(\xx(t))$ must have a limit when $t\to\infty$, let's say $L_{\infty}$. This implies that $\dot{L}(\xx(t)) \to 0$. Using first part, let be $\d_2>0$ such that for all $\xx$ in $B(\xx_*,\d_2)$, and for all $t>0$, $\phi(\xx,t) \in B(\xx_*,\epsilon)$. Now since $\phi(\xx,t)$ stay in a compact, there exists an accumulation point $\zz$ and a sequence $(t_n)_n$ growing to infinity such that $\lim_{n\to\infty} \phi(\xx,t_n) = \zz$. Because $\dot{L}$ is continuous, the limit $\lim_{n\to\infty} \dot{L}(\phi(\xx,t_n)) =0$ is actually $\dot{L}(\zz)$ and by hypothesis, $\zz$ must be $\xx_*$. Now because of the first part, as $\phi(\xx,t_n)$ approaches $\xx_*$, when $n$ is big enough $\phi(\xx,t)$ will stay arbitrarily near from $\xx_*$, assuring the convergence for all $t\to\infty$, and $\xx_*$ is attracting.

from the two last parts we conclude that $\xx_*$ is asymptotically stable.
\end{proof}

We now return to our original problem: test the stability of $(\Bstar,\Pstar)$, the fixed point of
\begin{equation} \label{eq:LV3}
    \begin{cases}
    \dotB &= F(B)(\a -\b G(\P) - \m F(B)) \\
    \dotP &= G(\P)(-\g +\d F(B) - \n G(\P))
    \end{cases}
\end{equation}
Literature tell us that there is no particular general method to find a Lyanupov function for generic equations. We note that concerning \prettyref{eq:LV}, we used a conserved quantity $V$ to prove periodicity of the trajectory. Let's try to use a similar version of this $V$ to obtain something that would not be constant and for our goal, decreasing :
\[W(B,\P) = \d \big(-P(B)F(\Bstar) + B\big) 
-\b\big(Q(\P) G(\Pstar)-\P\big).\]
We compute its derivative along a solution $(B,\P)$:
\begin{IEEEeqnarray*}{rCl}
\dot{W}(B,\P) 
&=& \frac{\dd}{\dd t} \bigg(\d \big(-P(B)F(\Bstar)+B\big)
-\b\big(Q(\P) G(\Pstar)-\P\big)\bigg) \\
&=& \d \big(-\frac{\dotB}{F(B)}F(\Bstar) + \dotB\big) 
-\b\big(\frac{\dotP}{G(\P)} G(\Pstar)-\dotP\big) \\
&=& \d \frac{\dotB}{F(B)}\big(-F(\Bstar) + F(B)\big) 
-\b\frac{\dotP}{G(\P)}\big( G(\Pstar)-G(\P)\big) \\
&=& \d (\a -\b G(\P) - \m F(B))\Delta_F 
+\b(-\g +\d F(B) - \n G(\P))\Delta_G \\
&=& \d (\m F(B_{**}) +\b G(\P_{**}) -\b G(\P) - \m F(B))\Delta_F 
+\b(-\d F(B_{**}) + \n G(\P_{**}) +\d F(B) - \n G(\P))\Delta_G \\
&=& \d (-\m\Delta_F - \b\Delta_G)\Delta_F 
+\b(\d\Delta_F-\n\Delta_G)\Delta_G \\
&=& -\d\m\Delta^2_F - \b\n\Delta^2_G
\end{IEEEeqnarray*}
with $\Delta_F= F(B)-F(\Bstar)$ and 
$\Delta_G= G(\P)-G(\Pstar)$. As a result, $\dot{W}(B,\P)$ is null only on $(\Bstar,\Pstar)$, otherwise it is strictly negative on $U=(\R^*_+)^2 \backslash\{(\Bstar,\Pstar)\}$. This is a Lyanupov function and thus, $(\Bstar,\Pstar)$ is asymptotically stable.  
\\ \\
For now, we don't know that if trajectories goes to $(\Bstar,\Pstar)$ independently of the initial point. We would like to prove a global convergence result. First we define the notion: 
\begin{definition}
    A fixed point $\xx_*$ of $\bdotx=\mathbf{F}(\mathbf{x})$ is said \emph{globally attractive} on a set $U$, if for all $\mathbf{x_0}\in U$, $\phi(\mathbf{x_0},t) \to \mathbf{x}_*$ as $t \to\infty$. In other words it is attractive without condition on the proximity of the initial point.
\end{definition}
\begin{definition}
    A fixed point $\xx_*$ of $\bdotx=\mathbf{F}(\mathbf{x})$ is said \emph{globally asymptotically stable} on a set $U$, if it is globally attractive and L-stable. In other words it is asymptotically stable without condition on the proximity of the initial point for the convergence.
\end{definition}
Now, if we look at our proof of \prettyref{th:Lianupov} on Lianupov functions, we see that the proximity that we needed to obtain convergence, was just in purpose to have a compact and invariant[to define] set. Without loss of generality we can take the $\d_2$ as big as we want, as long as $\phi(B(\xx_*,\d_2)\times\R_+)$ is bounded in $U$. A criteria on the boundedness of $L^{-1}([L(\xx_*),L_0])$ will help us to obtain global attractivity :
\begin{theoreme}
Under the suppositions of \prettyref{th:Lianupov}, if we have in addition that for a scalar $L_0>L(\xx_*)$,
\[ U_{L_0} = \{ \xx\in U | L(\xx) \leq L_0\} = L^{-1}\big([L(\xx_*);L_0]\big) \]
is bounded and whose closure is contained in $U$, then $\xx_*$ is globally asymptotically stable on $U_{L_0}$.
\end{theoreme}
\begin{proof}
First of all, $U_{L_0}$ is closed in the topology of $U$ because of the continuity of $L$. But since its euclidean closure is in U, it's actually closed in the all space. In addition, it is bounded, and thus, compact. Finally, all $\phi(\xx,0)$ in $U_{L_0}$ will stay in $U_{L_0}$, because of the monotonicity of $L(\phi(\xx,t))$ and we have the convergence by the same argument of \prettyref{th:Lianupov}, i.e. $L(\phi(\xx,t))\to L_\infty$, $\dot{L}(\phi(\xx,t))\to 0$, $\phi(\xx,t_n)\to \zz$ (because of the compacity of $U_{L_0}$), $\zz=\xx_*$, and $\phi(\xx,t)\to \xx_*$. This assure the global attractivity of $\xx_*$ on $U_{L_0}$, and hence, the global asymptotic stability, since the L-stability was assured yet.
\end{proof}

\begin{remarque}
If the hypothesis is true for all $L_0>L(\xx_*)$ then it is actually globally asymptotic stable everywhere on U. More generally, if a set $U_0\subset U$ (possibly $U$ itself), is invariant and each trajectory is bounded, the same result hold.
\end{remarque}
We recall that our Lyanupov function is 
\[W(B,\P) = \d \big(-P(B)F(\Bstar) + B\big) 
-\b\big(Q(\P) G(\Pstar)-\P\big)\]
and is defined in $(\R^*_+)^2$. For all $W_0>W(\Bstar,\Pstar)$, under the condition that \com{assumptions on $F$ and $G$ for divergence of $W$ when $B$ or $\P$ tends to o or $\infty$}, $W^{-1}([W(B_*,\P_*);W_0])$ must be bounded, and then $(B_*,\P_*)$ is globally asymptotically stable.