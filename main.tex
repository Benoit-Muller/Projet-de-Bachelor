\documentclass[a4paper]{report}
\usepackage{amsmath} %pour tous les trucs de math
\usepackage{systeme} %pour tous les systemes d'équations
%\usepackage{ bbold }  %pour toutes les doubles lettres
\usepackage{amssymb}  %pour les double Lettres
\usepackage{IEEEtrantools} %pour les équations en collonnes
\usepackage{amsthm} %pour les preuves
\usepackage[english]{babel} % la langue utilisée
\usepackage[utf8]{inputenc} % encodage symboles entrée
\usepackage[T1]{fontenc} % encodage symboles sortie
\usepackage{fancyhdr} %pour les entêtes et pied de page
%\usepackage[math]{blindtext} % pour le Lorem ipsum
%\usepackage{enumitem} %pour changer les listes
\usepackage[a4paper,textwidth=15.5cm]{geometry}
%\usepackage[framed,numbered]{mcode} %MatLab
\usepackage{graphicx} % pour les graphiques
%\usepackage{subfig} % pour les doubles figures
\usepackage{float} % pour bien positionner les figures
\usepackage[dvipsnames]{xcolor} % pour la couleur du texte et de la page
\usepackage{biblatex} % bibliographie
\usepackage{csquotes} % pour que la biblio s'adapte à la langue
\usepackage{prettyref}
\usepackage[hidelinks]{hyperref} % pour les hyperliens et références(mettre en dernier)

\newrefformat{fig}{Figure~[\ref{#1}]}
\newrefformat{it}{question~\ref{#1}.}
\newrefformat{eq}{(\ref{#1})}
\newrefformat{seq}{Section \ref{#1}}
\newrefformat{th}{Theorem \ref{#1}}

\newtheorem{theoreme}{Theorem} %[section]
\newtheorem{corollaire}{Corollary} %[theorem]
\newtheorem{lemme}{Lemma} %[theorem]
\theoremstyle{definition}
    \newtheorem{definition}{Definition} %[section]
\theoremstyle{remark}
     \newtheorem*{remarque}{Remark}

\renewcommand{\a}{\alpha}
\renewcommand{\b}{\beta}
\newcommand{\g}{\gamma}
\renewcommand{\d}{\delta}
\newcommand{\m}{\mu}
\newcommand{\n}{\nu}
\renewcommand{\P}{\Phi}
\newcommand{\dotx}{\dot{x}}
\newcommand{\bdotx}{\dot{\mathbf{x}}}
\newcommand{\xx}{\mathbf{x}}
\newcommand{\zz}{\mathbf{z}}
\newcommand{\doty}{\dot{y}}
\newcommand{\dotB}{\dot{B}}
\newcommand{\dotP}{\dot{\P}}
\newcommand{\Bstar}{B_{**}}
\newcommand{\Pstar}{\P_{**}}
\newcommand{\R}{\mathbb{R}}
\newcommand{\dd}{\mathrm{d}}

\newcommand{\com}[1]{\textcolor{ForestGreen}{[ \emph{#1} ]}}


\pagestyle{fancy}
\rhead{Chapter \thechapter}
\lhead{*Lotka–Volterra Equations*}
\chead{}

\title{*Lotka–Volterra Equations*}
\author{Benoît Müller}
\date{}

\addbibresource{biblio.bib}

\begin{document}
    \maketitle
    \tableofcontents
    
    \chapter{Linear Ordinary Equations}
\section{Solutions Space}
In this chapter,we study explicit autonomous ordinary equations of first order that are linear, namely equations of the form 
\[\dotxx=A\xx\]
where $A$ is a $n\times n$ matrix and $\xx$ is a differentiable function from a real interval to $\R^n$. Note that as its derivative must be $C^\infty$, $\xx$ must be $C^\infty$ to exist. 

First of all, we observe that the set of solution in a fixed interval is a vector subspace of $C^\infty$. Indeed for all solutions $\xx$, $\yy$ and a scalar $\a$, \[(\a\xx+\yy)'=\a\dotxx + \dot{\yy}=\a A\xx + A\yy=A(\a\xx+\yy)\] and the identical null function is trivially in the space. This motivate us to find a basis of this subspace and understand how to construct it.
\begin{definition}
    A collection of $k$ solutions $\xx_1,\dots,\xx_k$ on $I$ are said to be \emph{linearly independents} or \emph{independents}, if $\xx_1(t),\dots,\xx_k(t)$ are linearly independents for all $t\in I$.
\end{definition}
\begin{lemme}
A collection $\xx_1,\dots,\xx_k$ on $I$ of solutions are linearly independents if and only if there exists a moment when their positions are linearly independents.
\end{lemme}
\begin{proof}
If the solutions are independent we have the result by definition. In the other way, as $\xx\mapsto A\xx$ is Lipschitz continuous, we have by Picard that the solutions are uniques for a initial point given. Now if at a certain time $\tau$, the positions aren't independent, then we must have non all null scalar $\a_1,\dots,\a_k$ such that  $\a_1\xx_1(\tau) + \dots + \a_k\xx_k(\tau)=0$ but then 0 is a solution starting there and then actually  $\a_1\xx_1(t) + \dots + \a_k\xx_k(t)=0$ for all $t$ in $I$ and .
\end{proof}
This lemma prove that the space of solution has the same dimension the space $\R^n$ i.e. $n$.
In order to manipulate the solutions, we put them in a matrix, like $(\xx_1,\dots,\xx_k)$. In this form we see that we can actually extend the equation to matrix entries: 
\[\dot{X}=AX\]
and since 
\[\dot{X}=(\dotxx_1,\dots,\dotxx_k)  
\text{ and } 
AX=A(\xx_1,\dots,A\xx_k)=(A\xx_1,\dots,A\xx_k)\]
A matrix X is a solution if and only if its columns are vector solutions. We see that for a vector $v\in\R^k$, $Xv$ is a vector solution $(Xv)' = \dot{X}v = AXv$. This solution ca be written as $Xv = v_1\xx_1 + \dots + v_k\xx_k$ and is thus a linear combination of the column solutions of $X$. Thus a matrix solution permit us to easily write new solutions as linear combinaison of a collection of solutions. As the dimension of the space of solutions is $n$,exactly $n$ linear independents solutions will be enough to construct all solutions as a product of the matrix solution constructed with them.
\begin{definition}
    If a matrix solution $M=(\xx_1,\dots,A\xx_n)$ is square(k=n) and of full rank, then it is called a emph{fundamental matrix solution} and $\{\xx_1,\dots,A\xx_k\}$ a \emph{fundamental set of solutions}.
\end{definition}
with such a $M$, $M\R^n$ is the all space of solutions. At a time $t=0$, $M(0)v=v_1\xx_1(0) + \dots + v_k\xx_k(0)$ meaning that with the condition $\xx(0)=x_0$, v must be chosen to be the vector of coordinates of $x_0$ in the basis of the columns of M.
\\ \\
This done, we can look at the forms the solutions can take. In one dimension, the problem is $\dotx=ax$ and if a is non null, non trivial solutions respect $a=(\log|x|)'$ which give $\log|x(t)|=at+c$ and $x(t)=\pm e^{at+c}=x_0e^{at}$. We check easily that $e^{at}$ is indeed a solution. The $n$-dimensional case is more complex. We try to put the equation in its integral form and derive a property by recurrence (supposing $X_0=I$ for simplicity): 
\begin{IEEEeqnarray*}{rCl}
X(t) 
&=& I + \int_0^t AX(s) \dd s 
= I + \int_0^t A\bigg(I + \int_0^{s_1} AX(s_2) \dd s_2\bigg) \dd s_1
= I + tA +\int_0^t \int_0^{s_1} A^2X(s_2) \dd s_2 \dd s_1
\\ &=& I + tA + \frac12(tA)^2 + \int_0^t \int_0^{s_1} \int_0^{s_3} A^3X(s_3) \dd s_2 \dd s_1
\end{IEEEeqnarray*}
We see the Taylor expansion of $\exp$ appear, as a generalisation to matrices. This motivate the following definition:
\begin{definition}
If it converges, we define the \emph{exponential} $e^A$ of a square matrix A and the underlying function $\exp$ as the infinite sum 
\[e^A = \sum_{n=0}^\infty \frac{1}{n!}A^n\]
\end{definition}
\begin{lemme} \label{lem:exp}
The exponential of a matrix always converge
\end{lemme}
\begin{proof}
We use $\|.\|$ for the operator norm on matrices, keeping in mind that all matricies norms are equivalent. Now we have by basic properties of this norm that 
\[ \sum_{n=0}^N \|\frac{1}{n!}A^n\| 
\leq \sum_{n=0}^N \frac{1}{n!}\|A^n\| 
\leq \sum_{n=0}^N \frac{1}{n!}\|A\|^n \]

and this is the Taylor finite expansion of $\|A\|$, which converge for all value of $\|A\|$. As a result the sum is absolutely convergent for the operator norm, and then must converge.
\end{proof}
\begin{theoreme}
For a matrix $A$ and a scalar $t$, the quantity $e^{tA}$ is differentiable with respect to $t$ and has derivative $\dd/{\dd t} (e^{tA}) = Ae^{tA}$
\end{theoreme}
\begin{proof}
Each coordinate of the series is actually a convergent Taylor series in t and is then analytic. The theory of analytic functions tell us that they are $C^\infty$ and that we can derive term by term. However, we explain the argument of the proof in our case: As seen in \prettyref{lem:exp}, the sum converge absolutely, and in our case it converge locally uniformly in $t$ when $N\to\infty$, because if $|t|\leq t_{max}$, 
\[ \sum_{n=0}^N \|\frac{1}{n!}t^nA^n\|
\leq \sum_{n=0}^N \frac{1}{n!}(t_{max}\|A\|)^n\] converge without independence in $t$ for the last quantity. Similarily, \[ \frac{\dd}{\dd t}\sum_{n=0}^N\frac{1}{n!}t^nA^n
= \sum_{n=1}^N \frac{1}{(n-1)!}t^{n-1}A^n
= A\sum_{n=0}^N \frac{1}{n!}(tA)^n\]
converge locally uniformly to $Ae^{tA}$ for the same reason. As a result, since the partial sums and the derivatives of the partial sums converge locally uniformly, the limit is differentiable with derivative the limit of the derivatives of the partial sums, namely $Ae^{tA}$ and we have the result.
\end{proof}
All of this tell us that as we expected, $e^{tA}$ is a matrix solution to the linear differential equation.
\begin{corollaire}
The matrix $e^{tA}$ is a fundamental solution and each solution write $x(t)=e^{tA}x_0$
\end{corollaire}
\begin{proof}
We evaluate it in $t=0$ by direct calculation since the sum become finite: 
\[e^{0A}=\sum_{n=0}^\infty \frac{1}{n!}(0A)^n = I.\]
So the initial matrix is non singular, meaning that it will stay along the time non singular and $e^{tA}$ is a fundamental matrix solution with formula $x(t)=e^{tA}x_0$.
\end{proof}

For matrices that are not in a special form, we cannot explicitly compute the series of the exponential and then then neither the solutions. Specials forms of matrices whose exponential are commutable are for example the ones where the power of the matrix has a general formula, like diagonal matrices. We know that diagonal matrices are directly related to eigenvalues and eigenvectors. The eigenvector are the directions where the matrix act like in the one dimensional case and as often, it is a way to find similar results in the higher dimensional problems.

Let's investigate what happens in theses directions, and choose a candidate that look like a one dimensional solution,$e^{\l t}v$, for a real eigenvalue $\l$ of the matrix $A$ and the corresponding eigenvector~$\vv$. We compute its derivative and obtain 
\[\dd/{\dd t}(e^{\l t}v) =  e^{\l t}\l v = e^{\l t}Av = A(e^{\l t}v).\]
This give us indeed a non trivial solution since the eigenvector is non null. Theses kind of solutions can be seen in the computation of the exponential when we have a basis of eigenvectors and that in consequence, the matrix $A$ is diagonalisable like $A=PDP^{-1}$ where $D$ is the diagonal matrix of the eigenvalues and P is the non singular matrix formed by the eigenvector. Then 
\[ A^n= (PDP^{-1})^n = PDP^{-1}\cdots PDP^{-1} = PD^nP^{-1} \]
and 
\[ e^{tA} 
= Pe^{tD}P^{-1} 
= P\text{diag}(e^{\l_jt})P^{-1} 
= (v_1e^{\l_1t},\dots,v_ne^{\l_nt})P^{-1} 
\]
which is the fundamental matrix of the kind of solutions we find before, up to a reparametrization.

This was the simple case. More generally the matrix can have complex eigenvalues with complex eigenvectors, but the following lemma show us that complex solutions are made of real solutions :
\begin{lemme} \label{lem:complex}
If $\zz=\xx+i\yy$ is a complex solution of real and imaginary parts $\xx$ and $\yy$, then $\xx$ and $\yy$ are real solutions.
\end{lemme}
\begin{proof}
This can be shown by the simple fact that
\[\dotxx+i\dotyy =\dotzz = A\zz = A\xx+iA\yy \]
and that the real and respectively imaginary parts must be equals along the equalities, giving us $\dotxx=A\xx$ and $\dotyy=A\yy$ wich is the result.
\end{proof}
We must then consider complex solutions obtained by complex eigenvalues too. We sum up the results together in the following theorem.
\begin{theoreme} \label{th:eigensolutions}
For a complex eigenvalue $\s=\a+i\b$ of $A$ with eigenvector $w=u+iv$, we have the two independent solutions
\[e^{\a t}(\cos(\b t)u - \sin(\b t)v)\] 
\[e^{\a t}(\cos(\b t)v + \sin(\b t)u).\] 
In particular, if $\s$ is real, this give only one solution $e^{\a t}u.$ 
\end{theoreme}
\begin{proof}
First we have that $e^{\s t}w$ is a complex solution, since 
\[\ddt(e^{\s t}w) = e^{\s t}\s w = e^{\s t}Aw = A(e^{\s t}w).\]
Then by \prettyref{lem:complex}, the real and the imaginary parts are real solutions, we compute them by rewriting the complex solution:
\[e^{\s t}w = e^{\a t}(\cos(\b t) + i\sin(\b t))(u+iv)
= e^{\a t}(\cos(\b t)u - \sin(\b t)v) + ie^{\a t}(\cos(\b t)v + \sin(\b t)u).\]
We recognise the resulting solutions in the real and imaginary part and if we evaluate them in t=0, we get respectively $u$ and $v$. To be linearly dependent, $u$ should be proportional to $v$, i.e. $v=au$. But now $v$ cannot be null, and then $\b$ neither. As a result we can evaluate the solutions in $t=\pi/(2\b)$ and we obtain respectively $-e^{\a\pi/(2\b)}v$ and $e^{\a\pi/(2\b)}u$. These two points must be with the same proportion as in $t=0$. This is impossible as it would implies that $u=-av=-u=0$. The two solutions are then independents and the first part of the result is proven. It follow directly by setting $\b=0$ that the solutions is $e^{\a t}(\cos(0)u - \sin(0)v) = e^{\a t}u$ and  $e^{\a t}v$. But since the eigenvalue is real, the eigenvector is real too and $v=0$ letting only one non trivial solution.
\end{proof}
\begin{remarque}
Note that complex values come by pairs of conjugates, as well as the eigenvectors: \\
$Aw=\s w$ implies 
$$A\overline{w} = \overline{A}\overline{w} = \overline{Aw}= \overline{\s w} = \overline{\s}\overline{w}.$$
However, this doesn't give us a new solution because $$e^{\overline{\s} t}\overline{w}= \overline{e^{\s t}}\overline{w} = \overline{e^{\s t}w}$$
has the same real and imaginary part as $e^{\s t}w$ up to the sign. As a result, a complex eigenvalue give two solutions but together with its conjugate, they give us still two solutions, so we can find as many independent solutions as independent eigenvector we find, real or not.
\end{remarque}
\quad \\
Now we have to deal with the case when we don't have a basis of eigenvectors. In this case, some eigenvalue $\l$ with algebraic multiplicity $\m_a$ and a geometric multiplicity
$$\m_g=\dim\ker(A-\l I) < \m_a.$$
Such an eigenvalue is said \emph{defective}, and the matrix is said \emph{defective} too when it has at least one defective eigenvalue. We use here the concept of generalized eigenvector that come from the result about the Jordan form:

\begin{definition}
    A vector $\ww$ is a \emph{generalized eigenvector} of rank $m$  of a matrix A and corresponding to an eigenvalue $\l$, if it a vector (complex if $\l$ is complex) that satisfy
    $$(A-\l I)^m\ww=0 \quad\text{and}\quad (A-\l I)^{m-1}\ww\neq0$$
    for a $m\in\N^*$.
    
    A \emph{canonical} basis of generalised eigenvectors is a basis of generalised eigenvector such that for all generalized eigenvector $\ww$ of rank $m$ that is in the basis, for all $0<j<m$, $(A-\l I)^j\ww$ are generalised eigenvectors of rank $m-j$ with respect to $\l$, and are in the basis too.
\end{definition}

Note that with $m=1$, a generalised vector is a usual eigenvector.
    
\begin{theoreme}
    There always exist a canonical basis of generalised eigenvectors.
\end{theoreme}
\begin{proof}
    Without proof. See \com{some reference}.
\end{proof}

In term of computation, we start from $(A-\l I)\vv=0$ to find the eigenvectors. Then we search for some $\ww$ such that $(A-\l I)\ww=\vv$, assuring that $(A-\l I)^2\ww=(A-\l I)\vv=0$ and hence that $\ww$ is a generalised vector of rank 2. We continue like this making a chain of generalised vectors. Even if we decide to take into account complex vectors

Now we show how this is actually useful, by looking at the solution starting at a generalised eigenvector $\ww$ of rank $m$ with respect to the defective eigenvalue $\l$.
\begin{theoreme} \label{th:solutiondegeneree}
 For a canonical basis $B$ of generalised eigenvectors, we have a set of independent complex solutions with form $e^{t\l}p_\ww(B)$, where $\l$ is the eigenvalue assosiated to a generalised eigenvector $\ww$ of rank $m$ in the basis, and $p_\ww$ is a polynomial in $B[t]$ of degree $m-1$.
\end{theoreme}
\begin{proof}
We see easily that the basic property of the exponential that changes sum into product is true for the commutative matrices with the condition that the matrices commute. The proof is the same as in the real case as we have already proven absolute convergence. 
We write $A= \l I + (A-\l I) $ and $\l I$ is diagonal hence commutative with all matrices. Taking $\ww$ from a canonical basis, we get
\begin{IEEEeqnarray}{rCl} \label{eq:solutioncomplexe}
e^{tA}\ww 
&=& e^{t\l I + t(A-\l I)}\ww 
=e^{t\l I} e^{t(A-\l I)}\ww 
= e^{t\l}\sum_{n=0}^\infty \frac{1}{n!}t^n(A-\l I)^n\ww
= e^{t\l}\sum_{n=0}^{m-1} \frac{1}{n!}t^n\ww_n 
\end{IEEEeqnarray}
where the $\ww_n=(A-\l I)^n\ww$ are other generalised vectors in the canonical basis. So have indeed a polynomial in $t$ with generalised eigenvectors coefficients. These are surely independent solutions for different choices of $\ww$ since the generalised eigenvectors are supposed independent and they are the initial values of these solutions, the proof is complete.
\end{proof}
As a result with this theorem we obtain a independent set of $d$ complex solutions, where $d$ is the dimension of the space. But we know that eigenvalues and eigenvectors come by pairs of conjugates. It is the same for generalised eigenvectors and eigenvalues:
$$(A-\l I)^m\ww=0 \quad\text{and}\quad (A-\l I)^{m-1}\ww\neq0$$
implies
$$(A-\overline{\l} I)^m\overline{\ww}=0 \quad\text{and}\quad (A-\overline{\l} I)^{m-1}\overline{\ww}\neq0$$
So we will get solutions like \prettyref{eq:solutioncomplexe} by pairs of conjugates. by subtracting them or summing them, we get two real new solutions that we will call \emph{degenerated}:
$$
    e^{tA}\ww + e^{tA}\overline{\ww} 
    = e^{tA}(\ww+\overline{\ww})
    = 2e^{tA}\Re(\ww), 
$$
$$
    e^{tA}\ww - e^{tA}\overline{\ww} 
    = e^{tA}(\ww-\overline{\ww})
    = 2e^{tA}\Im(\ww).
$$
\com{prouver que on garde des solutions indépendantes}
\begin{corollaire} \label{cor:formesolutionlineaire}
    All solutions to linear system have coordinates that are linear combinations of the following functions :
    \begin{itemize}
    \item $e^{\l t}$ 
    \item $e^{\a t}\cos(\b t)$ , $e^{\a t}\sin(\b t)$
    \item $t^je^{\l t}$ , $t^je^{\a t}\cos(\b t)$ , $t^je^{\a t}\sin(\b t)$ 
    \end{itemize}
    where $\l$ is a real eigenvalue, $\s=\a+i\b$ is an imaginary eigenvalue, $0\leq j< m_a$ is a natural number with $m_a$ the algebraic multiplicity of $\s$. Note that each point is a generalisation of the precedent.
\end{corollaire}

\section{Stability}
In other terms, the \prettyref{th:eigensolutions} tell us that each non null real eigenvalue gives direction(s) where the trajectories are straight and of exponential velocity, each null eigenvalue give direction(s) where trajectories are fixed, and each non real eigenvalue gives subspace(s) (not lines) where the trajectories are like ellipses that change of size exponentially. The \prettyref{th:solutiondegeneree} add other sort of solutions as polynomials resized by a exponential.

All these considerations are only on the special directions. Depending of the sign of the real part of the eigenvalues, theses specials solutions go very fast to $0$, stay in an orbit, diverge very fast to infinity values, or maybee follow a polynomial. This motivate us to see how these specials solutions act together, what is the asymptotically comportment of trajectories, how stable is the origin, and make a classification about all theses factors.
First of all we define concepts about stability, and asymptotic converge. For this we put our self in a more general cadre which will be useful later, with a $C^1$ function $\FF$ and the equation 
\[\dotxx = \FF(\xx)\]
The regularity of $\FF$ give a flow $\phi(\xx_0,t)$ which encapsulate all solutions:
\begin{IEEEeqnarray*}{rCl}
\dot{\phi}(\xx_0,t) &=& \FF(\phi(\xx_0,t)) \\
\phi(\xx_0,0) &=& \xx_0
\end{IEEEeqnarray*}
\begin{definition}
    A solution x is \emph{Lyanupov stable} ( or \emph{L-stable}, or simply \emph{stable}) if the flow is continuous in $\xx_0$ and uniformly in $t$. Namely, if for all $\e>0$ there exist a $\d>0$ such that $\|\zz-\xx(0)\|<\d$ implies that for all $t\geq0$, $\|\phi(\zz,t)-\xx(t)\|<\e$.
\end{definition}
\begin{definition}
    Two solutions $\xx$ and $\yy$ are $\omega$\emph{-attracted} to each other if $\lim_{t\to\infty}\|\yy(t)-\xx(t)\| = 0$. The resulting equivalence class is called the \emph{basis of attraction}. A solution $\xx$ is said \emph{$\omega$-attracting} or \emph{attracting} if there is a $\d>0$ such that $\|\zz_0-\xx_0\|<0$ implies that $\phi(\zz_0,t)$ is in the basin of attraction of $\xx$. A solution is said \emph{globally $\omega$-attracting} (or \emph{globally attracting}) on a set if this set is in the basin of attraction. We do not need to specify the set if it's the all solution space.
\end{definition}
\begin{definition}
    A solution is \emph{asymptotically stable} if it is L-stable and attracting. It is said \emph{globally asymptotically stable} if it is L-stable and globally attracting
\end{definition}
\begin{lemme}
    Attractivity and L-stability are not consequence of the other in any sense. 
\end{lemme}
\begin{proof}
Any solution of $\dotxx=0$ is L-stable but any of them are attracting. We can construct a flow on the plane where all solutions tends to $(1,1)$, but all solutions that start in $\R\times\R_+^*$, even in the neighbourhood of $(1,1)$, goes around (0,0) before. \com{présenter la construction}
\end{proof}
\begin{lemme}
    When $\FF$ is linear, i.e. $\dotxx=A\xx$, all solutions have the same L-stability and attractivity.
\end{lemme}
\begin{proof}
In the linear case, $\phi$ is linear in the first variable, indeed $\phi(\xx_0,t) = e^{tA}\xx_0 =: X(t)\xx_0$. Now for any solution $\xx$ and all initial point $\zz$,
\[ \|\phi(\zz,t)-\xx(t)\| = \|X(t)\zz-X(t)\xx(0)\| = \|X(t)(\zz-\xx_0)\| =\|\phi((\zz-\xx_0),t)-\phi(0,t)\|, \] meaning that L-stability and attractivity is entirely determined by the stability of the trivial solution $\phi(0,t)=0$.
\end{proof}
\begin{remarque}
    Therefore, we can speak of the L-stability and the attractivity of a linear system, meaning that it applies to all solutions, or none of them, and doing it by looking at the trivial solution 0. In this case the system is L-stable if and only if there exists a $\d>0$ such that $\|\xx_0\|<\d$ implies that for all $t\geq0$, $\|\phi(\xx_0,t)\|<\e$. The system is attracting if and only if there exists a $\d>0$ such that $\|\xx_0\|<\d$ implies that $\phi(\xx_0,t) \to 0$.
\end{remarque}
 
 \begin{theoreme} \label{th:stablecondition}
     The linear system is L-stable if and only if each of its solutions is bounded for positive times.
 \end{theoreme}
 \begin{proof}
  Suppose the system is L-stable, and for contradiction that a solution $\xx$ is not bounded. Let $\d>0$ be the distance given by the stability, such that all solutions that start with a norm smaller than $\d$ don't go away the unit ball. We can then define an other solution $\yy=\d\xx/\|\xx(0)\|$ and $\|\yy(0)\|=\d$, so $\yy$ wont go away the unit ball and is bounded. This contradict the fact that $\xx$ and $\yy$ are proportionals and all solutions must be bounded.
  
  Suppose now that all solutions are bounded. Then the columns of $X=e^{tA}$ the fundamental system are bounded, implying that actually the norm of X is bounded (all norm on finite dimensional spaces are equivalent). Now we get 
  \[ \|\phi(x_0,t)\|=\|X(t)x_0\| 
  \leq \|X(t)\| \|\xx_0\| 
  \leq \max_{t\geq0} \|X(t)\| \|\xx_0\| \]
  which is smaller than any positive $\e$ as soon as $\|\xx_0\|<\e/\max_{t\geq0} \|X(t)\|$ and give us the stability of the sytem.
 \end{proof}
\begin{theoreme}
    The linear system is globally asymptotically stable if and only if it is attracting.
\end{theoreme}
\begin{proof}
By definition, global asymptotic stability implies global attractivity and so in particular attractivity with any radius condition. In the other direction, if it is attracting with radius condition $\d$, such that when $\|\xx_0\|<\d$, $\|\phi(\xx_0,t)\|\to0$. Then any solution $\xx$ can be written $\xx=\|\xx(0)\|\yy/\d$ where $\yy=\d\xx/\|\xx(0)\|$ is a proportional solution that start with norm $\d\|\xx(0)\|/\|\xx(0)\|=\d$ and is small enough to converge to zero, implying that $\xx=\xx(0)\yy/\d$ converge to zero too. The system is globally attracting by arbitrarity of $\xx$. Now that all solutions are converging to zero, they are all bounded and by \prettyref{th:stablecondition}, we know that the system is actually stable. Both condition of stability and global attractivity are reunited, the system is globally asymptotically stable.
\end{proof}
\begin{theoreme}
    The linear system is 
    \begin{enumerate}
     \item L-stable if and only if all the eigenvalues of the matrix have non positive real parts, and all the one that are defective have negative real part.
    \item globally asymptotically stable if and only if all the eigenvalues of the matrix
    have negative real parts.
    \end{enumerate}
\end{theoreme}
\begin{proof} 
\quad\\
\begin{enumerate}
\item 
By \prettyref{th:stablecondition}, we just have to show that the conditions on the eigenvalues are equivalent to the fact that the solutions are bounded. The \prettyref{cor:formesolutionlineaire} told us the possible forms of all solutions. The functions $t^je^{\a t}\cos(\b t)$ , $t^je^{\a t}\sin(\b t)$ describe all the possibilities of linear combinations for the coordinates.

For both, if the eigenvalue $\s=\a+i\b$ is non defective, then $j=0$ and they are bounded if $\a$ is non positive. If $\s$ is defective, $j>0$ and they are bounded if $\a$ is negative because the exponential is $O(t^j)$ for all $j$'s. The sytem is now stable.

Alternatively, if there exists an eigenvalue $\s=\a+i\b$ with eigenvector $w=u+iv$ that have a positive real part, then by \prettyref{th:eigensolutions}, there exist a solution $e^{\a t}(cos(\b t)u-\sin(\b t)v)$ and its norm $e^{\a t}\|cos(\b t)u-\sin(\b t)v\|$
is not bounded since $cos(\b t)u-\sin(\b t)v$ doesn't converge to zero. If $\s$ is defective and is purely imaginary, then \prettyref{th:solutiondegeneree} tell us that there exist a solution $p_w(t)$, polynomial of non null degree, and $p_w(t)\to\infty$ when $t\to\infty$ because of the dominant term. There exists solutions that are unbounded , and the system is not stable.

\item 
By \prettyref{th:stablecondition}, we just have to show that the conditions on the eigenvalues are equivalent to the fact that the system is attracting.

But following the considerations of the first part, if all eigenvalues have negative real parts, there is always a $e^{\a t}$ with $\a<0$ term in front of the bases solutions and they all converge to zero. The system is attracting, and globally asymptotically stable.

Alternatively if it's not the case and that $\s=\a+i\b$ has a non negative part, we have a solution $e^{\a t}p_w(t)$ that repect $\|e^{\a t}p_w(t)\| \geq \|p_w(t)\|$ which doesn't go to zero. The system is not attracting and then not globally asymptotically stable.
\end{enumerate}
\end{proof}

This is the result we wanted in this section. It motivate the categorisation of linear systems, by taking to account the nature of their stability. We make the difference when there is a rotation effect or not
\begin{definition}
    For a sytem $\dotxx=A\xx$, we set the following denomination of the matrix $A$ and more generally of the system, regarding the nature of the eigenvalues of $A$.
    \begin{itemize}
    \item \emph{Stable} : all eigenvalues are real, negative and non defective. ($\R^*_+$)
    \com{revoir notation pour ne pas confondre stable et L-stable\dots}
    \item \emph{Source (unstable)} : $-A$ is stable. i.e. all eigenvalues are real, positive, and non defective. ($\R^*_-$)
    \item \emph{elliptic (center)} : all eigenvalues are non null, purely imaginary, and non defective. ($i\R^*$)
    \item \emph{stable focus (sink)} : all eigenvalues are non null, with negative real part, and non defective. ($\R^*_-+i\R$)
    \item \emph{saddle} : there exists a real and positive eigenvalue, and a real and negative one, both non defective.
    \item \emph{hyperbolic} : stable, source, or saddle. i.e. all eigenvalues are real, non null, and non defective.
    \item \emph{degenerated} : there exists a defective eigenvalue.
    \end{itemize}
\end{definition}
\com{graphiques et présentations des structures possibles des solutions}
\begin{corollaire}
\quad
\begin{itemize} 
    \item Stables, elliptics, and stables focus linear systems are L-stable.
    \item Sources, saddles linear systems are not L-stable.
    \item Stable and stable focus linear systems are globally asymptotically stable.
    \end{itemize} 
\end{corollaire}
    \chapter{Lokta-Voltera equations and considerations on the affect of each species}

In this chapter, we present a model for the evolution of the population between preys and predators. For each of the predator and the prey population we define a function that quantify the size of the population with respect to time. The goal is to motivate a choice of differential equations that will describe the interaction and give a possible evolution of the two populations. Lotka-Volterra equations are well known equation in mathematical biology. To get straigth to the point, the equations are
\[\dot{x}=x(\a-\b y) \quad  \dot{y} = y(-\g + \d x).\]
where $x$ and $y$ represent the size of the population of preys, and predators respectively. The parameters $\a,\b,\g,\d$ are positive scalars. In the first chapter we presented a pragmatic and mathematical analysis of the linear system. Here, we propose first speak about the qualitative understanding of the equation.We will progressively motivate the choice behind this modelisation by showing equations related to it, and then assert a modification on the equations that will give us more possible euquations but still give the same study as Lotka-Volterra.

\section{Motivation} \label{sec:motiv}
We go back to the basics and remind that for a function $x$ of one variable $t$, 
\[x'(t) = \frac{\dd}{\dd t}x(t) = \dotx(t) \lim_{h\to0}\frac{x(t+h)-x(t)}{h}\] 
is the derivative when the limit exist. Qualitatively it indicate the amount of change of the function with respect to the variable. This amount is absolute with respect to the size of $x$ and doesn't depend relatively on it. In consequence, when $x$ is non null which is an assumption we will always make for initial value, $\frac{\dot{x}}{x}$ quantify the relative rate of change of the function. This could be in our situation the mean number of descendant of an individual. For example if a specie reproduce itself always with the same speed no matter the population or the environment, we can say that their grow rate is a constant c:
\[ \dot{x} = cx, \]
If the population is positive, 
\[ \frac{\dot{x}}{x} = \frac{d}{dt}(\log{x}) = c \]
and then by integrating,
\[ \log{x(t)} = ct + \log{x(0)}, \]
giving us $x(t) = x(0)e^{ct}$, the exponential growth. This is quite intuitive, if a population double each step time the general formula is of powers of two, the population wil just grow indefinitely. Alternatively if $c$ is taken negative, it mean we have loss of individuals, and the population decrease.

Obviously this is doesn't encapsulate the reality as the function grow very fast forever. The growth rate must decrease as the population increase. This come from multiple complex reasons such as environment capacities in food, space etc\dots For now we suppose it is from the simplest form, a linear decrease of this growth rate $\a -\b x$: 
\[ \dot{x} = x(\a -\b x) \]
That give us the logistic equation, where $\a$ represent the initial growth rate, and $\b$ how fast the growth rate slow down as the population size increase. Here we have one non trivial equilibrium when $\dot{x(t)}=0$ i.e. when $x(t)=\a/\b=x_*$. If not, we remark that $\dotx > 0$ when $ x < x_*$, and  $\dotx < 0$ when $ x < x_*$, meaning that $x_*$ seems stable. indeed we do the computations :
\begin{IEEEeqnarray*}{rCl}
   \a = \dotx\frac{x_*}{x(x_*-x)}
   = \frac{\dotx}{x} + \frac{\dotx}{x_* - x}
   = (\log x + \log|x_* - x|)'.
\end{IEEEeqnarray*}
By integrating from 0 to t we get
\[ \a t  = \log(x(t)) - \log|x_* -x(t)| -( \log(x_0) - \log|x_* -x_0|)
= \log \frac{x(t)(x_* -x_0)}{x_0(x_*-x(t))} \]
and rearranging terms
\[x(t) = \frac{x_0x_*e^{\a t}}{x_* + x_0(e^{\a t}-1)} \]
which is well defined for $t\in\R$ if $0<x_0<x_*$, and for $t\in[1/\a \log(1 - x_*/x_0),\infty]$ if $0<x_*<x_0$. We see that in both cases $x(t)\to x_*$ as $t\to\infty$.\com{graphique}

In conclusion for this logistic growth, the population will always stabilise in the direction of a unique non trivial equilibrium.

Now we want to introduce a second specie, the predator which alter the growth of the prey population, the same way we made the size of the population itself affect the rate of change. That mean that the growth rate will decrease in function of the prey, let's say linearly :
\[ \dotx = x(\a -\b y).\]
In the opposite, the growth rate of the predator increase together with the population. But it decrease without them :
\[ \doty = x(-\g +\d x).\]
This is the Lokta-Volterra equations for a prey-predator system.
When we added the affect of the other species, with didn't keep the affect of the population on itself. We can consider both of these affects and add a term in each growth rate:
  \begin{equation*}
    \begin{cases}
    \dotx &= x(\a -\b y - \m x) \\
    \doty &= y(-\g +\d x - \n y)
    \end{cases}
    \end{equation*}
Here we are, we obtained the equations we talked about. Let's talk deeply about them and about the possible modifications:

The idea is that we want to be able to modify the way the size of the species affect the rate. Traditional Lokta-Volterra equation consider the rate of growth as linear with respect to the populations, let's change that. We use functions $F$ and $G$ that are strictly monotonic,hence bijective, and increase from $0$ to $\infty$ to replace the linearity:
\begin{equation} \label{eq:LV}
    \begin{cases}
    \dotB = F(B)(\a -\b G(\P)) \\
    \dotP = G(\P)(-\g +\d F(B))
    \end{cases}
\end{equation}
Here again, $\a,\b,\g,\d$ are positive scalars.
Let's consider a real case where these considerations are meaningful:

The paper \cite{Gav} propose marine phages and bacteria. In this case bacteria doesn't seem to be limited by the environment and their size. We can suppose the affect of itself as negligible against the affect of the phages. With the modification, the effective and the physical size of the phage population are not the same. It seem that one of them quantify as the power p of the other. The reason behind this choice is that in the traditional Lotka-Volterra equations, we assume one predator meets one prey at a time. Here laboratory tests tend to say that the important meetings are when two (or maybe three) phages meet a bacteria.

This consideration make the power of $p=2$ appear on the value of the page population in the equation. Indeed we first wanted to wrote $\dotP=\P(-\g+\d B)$ meaning that the amount of change is the interactions between phages and bacteria, something like phages$\times$bacteria. But we need two phages, so something like phages$\times$phages$\times$bacteria which gives $\dotP=\P^2(-\g+\d B)$. A similar thinking motivate the writing of $\dotB=B(\a-\b \P^2)$. In other words, we can say that they are hunting  teams of $p$ phages. For the mathematical study, we will keep general functions F and G that we can change in function of these kind of considerations on the affect.

\section{Mathematical study}
We have a non trivial equilibrium where $\dotB=\dotP=0$ :
\[ B_* = F^{-1}(\g/\d) \quad \P_* = G^{-1}(\a/\b)\]
Note that if one of the two populations is constant, then the other must be constant too and they must be in this total equilibrium. Let's rewrite \prettyref{eq:LV} using this notation :
\begin{equation} \label{eq:LV*}
    \begin{cases}
    \dotB = \b F(B)(G(\P_*) - G(\P)) \\
    \dotP = \d G(\P)(-F(B_*) + F(B))
    \end{cases}
\end{equation}
Now we can study the sign of the derivatives $\dotB$ and $\dotP$ and draw a phase plane. The positive values $B_*$ and $\P_*$ divide the positive plane $\R_+^2$ in four regions.
\com{phase plane}
Trajectories seem to turn around the center of equilibrium and we search for a first integral by taking the cross product of \prettyref{eq:LV*} and dividing by $F(B)G(\P)$ :
\begin{IEEEeqnarray*}{rCl} 
    0 &=& \big(\dotB \d G(\P)(-F(B_*) + F(B))
        - \dotP \b F(B)(G(\P_*) - G(\P)) \big) \frac{1}{F(B)G(\P)} 
        \IEEEyesnumber \label{eq:1integral} \\
    &=& \d(-\frac{\dotB}{F(B)}F(B_*) + \dotB)
    -  \b(\frac{\dotP}{G(\P)}G(\P_*) - \dotP) \\
    &=& \Big(\d (-P(B)F(B_*) + B) -\b(Q(\P) G(\P_*))-\P)\Big)'
\end{IEEEeqnarray*}
where $P$ and $Q$ are primitives of $1/F$ and $1/G$, and exist because $F$ and $G$ are continuous. This give us a conserved quantity 
\begin{IEEEeqnarray*}{rCl}
V(B,\P) &=& \d \big(-P(B)F(B_*) + B\big) 
-\b\big(Q(\P) G(\P_*)-\P\big) \\
&=& V(B(0),\P(0)) \\
&=:& V_0
\end{IEEEeqnarray*}
Now we know then that $(B,\P) \in V^{-1}(\{V_0\})$, a closed set as V is continuous. To understand the level set, we compute the gradient :
$$\nabla V(B,\P)
= \big(\d( -\frac{F(B_*)}{F(B)} + 1) , 
-\b(\frac{G(\P_*)}{G(\P)}-1)\big)^\top.$$
It never vanishes, except in the equilibrium and is continuous. As a result, the level set doesn't have interior otherwise the function would be constant and the gradient null. Since we have the local unicity of the solution, the level set is actually a closed line, \ie the solutions are in an closed orbit.
\com{plot $\to$ \url{https://www.desmos.com/calculator/nqxtcrysok}} .
Now that we know that it's cyclic with a period $\tau$, we derive a modified Voltera principle by integrating these qualities obtained from \prettyref{eq:LV*} :
\[ \dotB / F(B) = \b (G(\P_*) - G(\P)). \]
the left side gives
\begin{IEEEeqnarray*}{rCl}
    \int_0^\tau \frac{\dotB}{F(B)} 
     &=& \int_0^\tau (P(B))'
     = P(B(\tau))-P(B(0)) =  P(B(0))-P(B(0)) =0,
\end{IEEEeqnarray*}
and then the rigth side
  \[ 0= \int_0^\tau \b (G(\P_*) - G(\P))
    = \tau\b G(\P_*) - \int_0^\tau \b  G(\P) \]
which implies that
 \[  G(\P_*) = \frac1\tau\int_0^\tau G(\P). \]
 Note that if $G=$Id, this says that the mean of the population $\P$ during time is $F(B_*)$. By the same argument with the second equation, we obtain that similarly
  \[  F(B_*) = \frac1\tau\int_0^\tau F(B). \]
  \\ \\
  In \prettyref{sec:motiv}, we explained how the own size of a population can affect its growth. We presented the logistic growth consequence of this equation :
  \[\dotx = x(\a - \b x)\]
  We add then a term in the rates of \prettyref{eq:LV} to modelise the fact that the growth rate of a population decrease with the size of population due to environment capabilities or competition. Again we use $F$ and $G$ to quantify their importance :
  \begin{equation} \label{eq:LV2}
    \begin{cases}
    \dotB &= F(B)(\a -\b G(\P) - \m F(B)) \\
    \dotP &= G(\P)(-\g +\d F(B) - \n G(\P))
    \end{cases}
\end{equation}
with new positive scalars $\m$ and $\n$. We search for non trivial equilibrium where $\dotB_{**}=\dotP_{**}=0$ and obtain a linear system in $F(B_{**})$ and $G(\P_{**})$ :
\begin{equation*}
    \begin{cases}
    -\a = - \m F(B_{**}) -\b G(\P_{**}) \\
    \g = \d F(B_{**}) - \n G(\P_{**}) 
    \end{cases}
\end{equation*}
which give
\begin{IEEEeqnarray*}{rCl}
    \m\g-\d\a &=& \m\d F(B_{**}) - \m\n G(\P_{**}) - \d\m F(B_{**}) -\d\b G(\P_{**}) \\  &=& (-\m\n-\d\b) G(\P_{**}) 
\end{IEEEeqnarray*}
and then as $\m\n+\d\b > 0$ and supposing $\a\d>\g\m$
\begin{IEEEeqnarray}{rCl} \label{eq:Pstar}
G(\P_{**}) = \frac{\a\d-\g\m}{\b\d+\n\m} ,\quad
\P_{**}=P^{-1}\Big(\frac{\a\d-\g\m}{\b\d+\n\m}\Big)
\end{IEEEeqnarray}
Similarly,
\begin{IEEEeqnarray}{rCl} \label{eq:Bstar}
F(B_{**}) = \frac{\b\g+\n\a}{\b\d+\n\m}, \quad
B_{**} = F^{-1} \Big(\frac{\b\g+\n\a}{\b\d+\n\m}\Big)
\end{IEEEeqnarray}
Here we cannot derive a first integral like we did in \prettyref{eq:1integral} by separating variables $B$ and $\P$. Instead, we want to test the stability of $(\Bstar,\Pstar)$. First we can try to use the theorem of linearization to check the possible asymptotic stability of the system.
\begin{IEEEeqnarray*}{C}
D_{B,\P} \begin{pmatrix}
    F(B)(\a -\b G(\P) - \m F(B))\\
    G(\P)(-\g +\d F(B) - \n G(\P))
\end{pmatrix} \\
= \begin{pmatrix}
    F'(B)(\a -\b G(\P) - \m F(B)) - F(B)\m F'(B)
    & -F(B)\b G'(\P)
    \\
    G(\P)\d F'(B)
    & G'(\P)(-\g +\d F(B) - \n G(\P))-G(\P)\n G'(\P)
\end{pmatrix} 
\end{IEEEeqnarray*}
We evaluate in the equilibrium and obtain
$$ \begin{pmatrix}
     - F(\Bstar)\m F'(\Bstar) & -F(\Bstar)\b G'(\Pstar)
    \\
    G(\Pstar)\d F'(\Bstar)   & -G(\Pstar)\n G'(\Pstar)
\end{pmatrix} $$

Now we could replace $F(\Bstar)$ and $G(\Pstar)$ by their expression \prettyref{eq:Bstar} and \prettyref{eq:Pstar}, and compute the characteristic polynomial. But we would still have the the expressions $F'(\Bstar)$ and $G'(\Pstar)$, that we know to be positive but nothing else. The polynomial isn't in a simple form and the computation of the eigenvalues become excessively cumbersome, because of the number of parameters and the hypothesis on them. More simply we show that a $2\times2$ matrix with the same signs of this one is always stable. Indeed all the factors are positives and we can write the matrix like
$$\begin{pmatrix}-a&-b\\c&-d\end{pmatrix}$$
with characteristic polynomial $\l^2+(a+d)\l+ad+bc$ of discriminant $(a+d)^2-4(ad+bc)$. If the discriminant is negative, then the real part of the root will be $-(a+d)/2<0$, and else $$\frac{-(a+d)\pm\sqrt{(a+d)^2-4(ad+bc)}}{2} 
< \frac{-(a+d) \pm (a+d) }{2} \leq 0$$
So the eigenvalues are stable and the linearized system too. By the theorem of linearization, the non-linear system is asymptotically stable in the equilibrium. In other words all solutions near enough this point tends to exponentially to it. However, we don't know how big is the basin of attraction.

\begin{comment}
 We propose here to fix things, we will check numerically for the example and see that even the numerical expression isn't simple. Let's say $F(B)=B$, $G(\P)=\P^2$ and fix arbitrary the parameters making attention to $\a\d>\g\m$. Let's say $\a=4$, $\b=5$, $\m=2$, $\g=1$, $\d=3$, $\n=6$.
We get then $\Bstar=F(\Bstar)
= (5\times1+6\times4)/(5\times3+6\times2)
= 29/27$, 
$F'(\Bstar)=1$, 
$\Pstar^2=G(\Pstar)
= (4\times3-1\times2)/(5\times3+6\times2)
= 10/27$, 
$G'(\Pstar)=2\Pstar=2\sqrt{10/27}$. The matrix is then
$$ \begin{pmatrix}
     - 29/27*2 & -29/27*5*2\sqrt{10/27}
    \\
    10/27*3   & -10/27*6*2\sqrt{10/27}
\end{pmatrix} 
= \begin{pmatrix}
    -58/27 & -290 \sqrt{10/3}/81 \\ 10/9 & -40\sqrt{10/3}/27
\end{pmatrix} 
$$
with the eigenvalues 
$$\l_1=\frac{1}{243} \big(-3 (87 + 20 \sqrt{30}) + 3 i \sqrt{3 (-6523 + 4060 \sqrt{30})}\big) 
\quad \text{ and } \quad \l_2=\bar{\l}_1.$$
We have then an stable linearized system
\end{comment}
For this we develop the theory of stability of Lianupov. Consider a differential equation $\bdotx=\mathbf{F}(\mathbf{x})$, such that there exist a unique solution for each initial point and for all $t\geq0$. Such solutions are denoted by the flow $\phi$, such that $t\mapsto \phi(\mathbf{x}_0,t)$ is the solution initialised at $\mathbf{x}_0$. We recall the following definitions

\begin{definition}
 A fixed point $\mathbf{x}_*$ of  is said \emph{Lyanupov stable} or \emph{L-stable} if 
 for all $\epsilon>0$, there exists a $\delta>0$ such that for all $\mathbf{x_0}$ and for all $t\geq0$, $\|\mathbf{x_0} - \mathbf{x}_*\| < \delta$ implies $\|\phi(\mathbf{x_0},t) - \mathbf{x}_*\| < \epsilon$.
\end{definition}

\begin{definition}
A fixed point $\mathbf{x}_*$ of  is said \emph{attracting} if there exists a $\delta>0$ s.t. for all $\mathbf{x_0}$, $\|\mathbf{x_0} - \mathbf{x}_*\| < \delta$ implies that $\phi(\mathbf{x_0},t) \to \mathbf{x}_*$ as $t \to\infty$.
\end{definition}

\begin{definition}
    A fixed point which is L-stable and attracting is said \emph{asymptoticaly stable}
\end{definition}

\begin{remarque}
We remind that have shown that these two notions are different in \prettyref{rem:stabilité}.
\end{remarque}

Now we can introduce a tool that will be useful to understand the limit comportment of the trajectories and will a tool to proove asymptotic stability.

\begin{definition}
    Assume $\mathbf{x}_*$ is a fixed point of a equation $\bdotx = \mathbf{F}(\mathbf{x})$, and let a function $L:U\to\R$ defined in a neighbourhood of $\mathbf{x}_*$.
    \\
    The function $L$ is called a \emph{weak Lyanupov function} if \[L(\mathbf{x})>L(\mathbf{x}_*)\] 
    and 
    \[ \dot{L}(\xx):=\frac{\dd}{\dd t}L(\phi(\mathbf{x},t))|_{t=0} \leq 0 \]
    for all $\mathbf{x}$ in $U$.
    \\
    The function $L$ is called a \emph{Lyanupov function} (or a \emph{strict Lyanupov function}) provided that we have the strict inequality $\dot{L}(\xx) < 0$ when $\xx\neq\xx_*$ (For this condition we say that $\dot{L}$ is semipositive definite in $\xx_*$ and it guarantee automatically $L(\mathbf{x})>L(\mathbf{x}_*)$.).
\end{definition}

\begin{theoreme} \label{th:Lianupov}
Let $\mathbf{x}_*$ a fixed point of the differential equation $\bdotx = \mathbf{F}(\mathbf{x})$ and $L$ a weak Lyanupov function on $U\ni\mathbf{x}_*$. Then $\mathbf{x}_*$ is L-stable. If $L$ is a strict lyanupov function, $\mathbf{x}_*$ is attracting too and thus $L$ is asymptotically stable.
\end{theoreme}
\begin{proof}
L-stable: Let's suppose first that $L$ is a weak Lyanupov function. Since $\dot{L}(\xx) \leq 0$, $L(\xx)$ is decreasing. Let be $\epsilon>0$. Up to taking it smaller, we suppose $B(\xx_*,\epsilon) \subset U$. But $L$ is continuous and $\partial B(\xx_*,\epsilon)$ is compact, we can then define
\[m=\min_{\partial B(\xx_*,\e)}{L} \; \geq \; L(\xx_*),\]
and since $L$ is continuous, there exists a $\epsilon>\d>0$ such that $L(\xx)<m$ when $\|\xx-\xx_*\|<\d$. Now, for all $\xx\in B(\xx_*,\d) \subset B(\xx_*,\epsilon)$, $L(\xx)<m$ mean that $\phi(x,t)$ cannot go out of $B(\xx_*,\epsilon)$, otherwise it would cross $\partial B(\xx_*,\epsilon)$ in a certain $t>0$, and  we would have the contradiction $m>L(\xx)=L(\phi(\xx,0)) \geq L(\phi(\xx,t)) \geq m$. Whave the L-stability.

Attracting when strict Lyanupov: Since $L(\xx)$ is decreasing and bounded over $L(\xx_*)$, $L(\xx(t))$ must have a limit when $t\to\infty$, let's say $L_{\infty}$. This implies that $\dot{L}(\xx(t)) \to 0$. Using first part, let be $\d_2>0$ such that for all $\xx$ in $B(\xx_*,\d_2)$, and for all $t>0$, $\phi(\xx,t) \in B(\xx_*,\epsilon)$. Now since $\phi(\xx,t)$ stay in a compact, there exists an accumulation point $\zz$ and a sequence $(t_n)_n$ growing to infinity such that $\lim_{n\to\infty} \phi(\xx,t_n) = \zz$. Because $\dot{L}$ is continuous, the limit $\lim_{n\to\infty} \dot{L}(\phi(\xx,t_n)) =0$ is actually $\dot{L}(\zz)$ and by hypothesis, $\zz$ must be $\xx_*$. Now because of the first part, as $\phi(\xx,t_n)$ approaches $\xx_*=\zz$, when $n$ is big enough $\phi(\xx,t)$ will be arbitrarily near from $\xx_*$ for all $t>t_n$, assuring the convergence for all $t\to\infty$, and $\xx_*$ is attracting.

from the two last parts we conclude that $\xx_*$ is asymptotically stable.
\end{proof}

We now return to our original problem: test the stability of $(\Bstar,\Pstar)$, the fixed point of
\begin{equation} \label{eq:LV3}
    \begin{cases}
    \dotB &= F(B)(\a -\b G(\P) - \m F(B)) \\
    \dotP &= G(\P)(-\g +\d F(B) - \n G(\P))
    \end{cases}
\end{equation}
Literature tell us that there is no particular general method to find a Lyanupov function for generic equations. We note that concerning \prettyref{eq:LV}, we used a conserved quantity $V$ to prove periodicity of the trajectory. Let's try to use a similar version of this $V$ to obtain something that would not be constant and for our goal, decreasing :
\[W(B,\P) = \d \big(-P(B)F(\Bstar) + B\big) 
-\b\big(Q(\P) G(\Pstar)-\P\big).\]
We compute its derivative along a solution $(B,\P)$:
\begin{IEEEeqnarray*}{rCl}
\dot{W}(B,\P) 
&=& \frac{\dd}{\dd t} \bigg(\d \big(-P(B)F(\Bstar)+B\big)
-\b\big(Q(\P) G(\Pstar)-\P\big)\bigg) \\
&=& \d \big(-\frac{\dotB}{F(B)}F(\Bstar) + \dotB\big) 
-\b\big(\frac{\dotP}{G(\P)} G(\Pstar)-\dotP\big) \\
&=& \d \frac{\dotB}{F(B)}\big(-F(\Bstar) + F(B)\big) 
-\b\frac{\dotP}{G(\P)}\big( G(\Pstar)-G(\P)\big) \\
&=& \d (\a -\b G(\P) - \m F(B))\Delta_F 
+\b(-\g +\d F(B) - \n G(\P))\Delta_G \\
&=& \d (\m F(B_{**}) +\b G(\P_{**}) -\b G(\P) - \m F(B))\Delta_F 
+\b(-\d F(B_{**}) + \n G(\P_{**}) +\d F(B) - \n G(\P))\Delta_G \\
&=& \d (-\m\Delta_F - \b\Delta_G)\Delta_F 
+\b(\d\Delta_F-\n\Delta_G)\Delta_G \\
&=& -\d\m\Delta^2_F - \b\n\Delta^2_G
\end{IEEEeqnarray*}
with $\Delta_F= F(B)-F(\Bstar)$ and 
$\Delta_G= G(\P)-G(\Pstar)$. As a result, $\dot{W}(B,\P)$ is null only on $(\Bstar,\Pstar)$, otherwise it is strictly negative on $U=(\R^*_+)^2 \backslash\{(\Bstar,\Pstar)\}$. This is a Lyanupov function and thus, $(\Bstar,\Pstar)$ is asymptotically stable. \com{graphic of W as potential energy}
\\ \\
For now, we don't know that if trajectories goes to $(\Bstar,\Pstar)$ independently of the initial point. We would like to prove a global convergence result. First we recall the notion: 
\begin{definition}
    A fixed point $\xx_*$ of $\bdotx=\mathbf{F}(\mathbf{x})$ is said \emph{globally attractive} on a set $U$, if for all $\mathbf{x_0}\in U$, $\phi(\mathbf{x_0},t) \to \mathbf{x}_*$ as $t \to\infty$. In other words it is attractive without condition on the proximity of the initial point.
\end{definition}
\begin{definition}
    A fixed point $\xx_*$ of $\bdotx=\mathbf{F}(\mathbf{x})$ is said \emph{globally asymptotically stable} on a set $U$, if it is globally attractive and L-stable. In other words it is asymptotically stable without condition on the proximity of the initial point for the convergence.
\end{definition}
Now, if we look at our proof of \prettyref{th:Lianupov} on Lianupov functions, we see that the proximity that we needed to obtain convergence, was just in purpose to have a compact and invariant set. Without loss of generality we can take the $\d_2$ as big as we want, as long as $\phi(B(\xx_*,\d_2)\times\R_+)$ is bounded in $U$. A criteria on the boundedness of $L^{-1}([L(\xx_*),L_0])$ will help us to obtain global attractivity :
\begin{corollaire}
Under the suppositions of \prettyref{th:Lianupov}, if we have in addition that for a scalar $L_0>L(\xx_*)$,
\[ U_{L_0} = \{ \xx\in U | L(\xx) \leq L_0\} = L^{-1}\big([L(\xx_*);L_0]\big) \]
is bounded and whose closure is contained in $U$, then $\xx_*$ is globally asymptotically stable on $U_{L_0}$.
\end{corollaire}
\begin{proof}
First of all, $U_{L_0}$ is closed in the topology of $U$ because of the continuity of $L$. But since its euclidean closure is in U, it's actually closed in the all space. In addition, it is bounded, and thus, compact. Finally, all $\phi(\xx,0)$ in $U_{L_0}$ will stay in $U_{L_0}$, because of the monotonicity of $L(\phi(\xx,t))$ and we have the convergence by the same argument of \prettyref{th:Lianupov}, i.e. $L(\phi(\xx,t))\to L_\infty$, $\dot{L}(\phi(\xx,t))\to 0$, $\phi(\xx,t_n)\to \zz$ (because of the compacity of $U_{L_0}$), $\zz=\xx_*$, and $\phi(\xx,t)\to \xx_*$. This assure the global attractivity of $\xx_*$ on $U_{L_0}$, and hence, the global asymptotic stability, since the L-stability was assured yet.
\end{proof}

\begin{remarque}
If the hypothesis is true for all $L_0>L(\xx_*)$ then it is actually globally asymptotic stable everywhere on U. More generally, if a set $U_0\subset U$ (possibly $U$ itself), is invariant and each trajectory is bounded, the same result hold.
\end{remarque}
We recall that our Lyanupov function is 
\[W(B,\P) = \d \big(-P(B)F(\Bstar) + B\big) 
-\b\big(Q(\P) G(\Pstar)-\P\big)\]
and is defined in $(\R^*_+)^2$. For all $W_0>W(\Bstar,\Pstar)$, under the condition that \com{assumptions on $F$ and $G$ for divergence of $W$ when $B$ or $\P$ tends to o or $\infty$}, $W$ tends to infinity when $B$ or $\P$ tends to infinity, so $W^{-1}([W(B_*,\P_*);W_0])$ must be bounded, and then $(B_*,\P_*)$ is globally asymptotically stable on the positive quadrant.
    \chapter{Modelisation about Covid-19}
In the last chapter, we gained a good intuition how and why \LV equations model population interactions. We have seen what are the effects of adding terms in the growth rate of a equation. We recall that the general understanding of growth rate of a scalar function $x_j$ is $\dot{x}_j/x_j$ when $x_j\neq0$. This is mostly interesting when we have to deal with a equation of the form $\dot{x}_j=x_jf(x)$, because the growth rate is just $f(x)$. We would like to use these ideas of the second chapter to elaborate an equation that would simulate a pandemic. We do not have the audacity to say that the model will be a perfect modelisation of the problem and that it will even predict the future. It's actually not the goal, our main goal is to learn how to create a differential equation so that it fit the kind of comportment we need. The biological context is a pretext, the definition of a particular situation, which determine the qualitative considerations that our final solution will satisfy.
\section{Quantities and considerations}
First we present the quantities we will have to deal with and link together with differential equations. We suppose we have a population of a size $P$ that will be confronted to a virus. The general idea is that the virus is the predator and the people are the prey. We have a number $V$ of vulnerable people that have never been infected. We could have set a variable for virus but there is no such idea of quantity of virus, so we consider that infected people become part of the predators, and we set a number $I$ of people that are infected. As it is the case most of the time for virus, there is a phenomena of immunity (at least partial) that follow an infection. As a result, the number $R$ of resistant people is the number of people that have been infected in the past. We choose the case where infected people are mostly immediately infectious after infection, so $V$ is the number of infectious people too.

In conclusion we have a population $P$ where each person can successively belong to each of the three parts $V$, $I$, and $R$. In other words we have $V+I+R=P$ and the scheme is $V\to I\to R$.
As the number of people is supposed to be big, we can use real variable for all these quantities even if the populations are integers, and we make them vary along the free variable of time.

\section{The models}
\subsection{A \LV approach} \label{sec:premier-modèle}
We want to use the \LV model, just like if infected people where the predators of the vulnerable ones. The general form of this system is 
\begin{IEEEeqnarray*}{rCl}
    \dotV&=&V(\a+\b I) \\
    \dotI&=&I(\g+\d V).
\end{IEEEeqnarray*}
But what are the sign of the scalars? If there is no (-more) infected, $I=0$ and $0=\dotV=\a V$ implies that $\a=0$. The parameter $\b$ represent how the infected affect the vulnerable people. It's then negative because more infected implies less vulnerable. More precisely, if the mean number of contacts of a person is $\mu$ then the mean of infection by unit of time is $\mu$ times the proportion of infected, $I/P$. This give indeed a term $\mu/PI=-\b I$. For the second equality, the rate of infected contain a term for the people that add from the vulnerable group, $ -\b I$, and $\d=\b$. The other term models the people that are leaving the infected group for the resistant one. We can suppose that the period of infectivity is significantly smaller than the velocity of the pandemic, namely the change of size of the groups. So the number of people that leave the infected group at a time $t$ is the same as the number of people that was going into it a bit sooner. We can suppose that in average, a fraction of the infected group leave it and so it give a term $\g I$ with $\g<0$. With all these considerations we rename the paramters and take them all positive so we can write 
\begin{IEEEeqnarray}{rCl} \label{eq:LV-virus}
    \dotV&=&-\b VI \\
    \dotI&=&I(-\a+\b V).
\end{IEEEeqnarray}
We would like the resistant group to have the same oppostite rate as the term $-\a I$ that represent people leaving the infected group. So we have $\dotR=\a R$. We see that all of this seem coherent since $$(V+I+R)'=\dotV+\dotI+\dotR=-\b VI + I(-\a+\b V) + \a R =0$$
and the population is always conserved.

There is no isolated fixed point of the system \prettyref{eq:LV-virus}, but all the axis where $I=0$ is fixed. This show that once the virus is eradicate there is no more change. Here, the fixed axis $I=0$ will introduce a zero eigenvalue in the linearization, and we cannot use the theorem of linearization because we need stable eigenvalues. More than that, because the fixed points are not isolated, it implies that they are not asymptotically stable because there exists non convergent solutions that start as near as we want. Instead of this method, we will use first integrals, \ie constant quantities along time. We consider the system terminated if $I$ or $V$ reach zero. 
The function $V$ is decreasing, and $I$ change of direction when $V$ passes $\a/\b$. We try to obtain a first integral like in the second chapter by separating the variables. Supposing $I\neq0\neq V$, we take the cross product of 
\begin{IEEEeqnarray*}{rCl}
    \frac{\dotV}{V}&=&-\b I \\
   \frac{\dotI}{I}&=&-\a+\b V.
\end{IEEEeqnarray*}
and obtain 
$$0 = \frac{\dotV}{V}(-\a+\b V) + \b\dotI 
=  \dotV(-\frac{\a}{V}+\b) + \b\dotI
= (-\a\log(V)+\b V + \b I)',$$
so we have the constant quantity 
$$L(V):=-\a\log(V)+\b V + \b I = -\a\log(V_0)+\b V_0 + \b I_0$$
$$0=-\a\log(\frac{V}{V_0})+\b (V-V_0) + \b (I-I_0)$$
and an explicit form for the curve : 
\begin{equation} \label{eq:LV-virus-curve}
    I= \frac{\a}{\b}\log(\frac{V}{V_0})- (V-V_0) + I_0
= \frac{\a}{\b}\log(\frac{V}{V_0})- V + P.
\end{equation}
Since its derivative is negative, $V$ is decreasing and positive so it has a limit $V_\infty$. Taking the limit of the expression we just get when time goes to infinity, we see that since $I$ is bounded by $P$, $V$ cannot goes to zero, and $V_\infty>0$. For $I$, we see that $(I+V)'=-\a I\leq0$ so similarly the positive quantity $I+V$ must converge, and its derivative must converge to zero, so $I=(I+V)'/\a\to0$. As a result, we can now take the limit of \prettyref{eq:LV-virus-curve} and get
$$0= \frac{\a}{\b}\log(\frac{V_\infty}{V_0})- V_\infty + P.$$
We show that we always have a point $V_\infty$ such that this is satisfied. Indeed $L(V)-L(0)\to-\infty$ when $V\to0$ and $$L(P)-L(0) = \frac{\a}{\b}\log(\frac{P}{V_0})\geq\frac{\a}{\b}\log(1)=0.$$
Moreover, $$\frac{d}{dV}(L(V)-L(0))=\frac{\a}{\b}\frac{1}{V}- 1$$
changes sign only one time in $V=\frac{\a}{\b}$ where there is a max, so if there is two zeros, $P$ must be between them and there is exactly one zero such that $V_\infty\leq R$. The max of $I$ is
$$I_{max}=L(\a/\b)-L(0) = \frac{\a}{\b}(\log(\frac{\a}{\b V_0})-1) + P$$
All in all, we have a pandemic (the number of infected increase) if and only if $V_0>\a/\b$, and in this case the infected group increase to $I_{max}$, before to go down to zero, and there will be $V_\infty>0$ people that will never have been infected. \com{plot : \url{https://www.desmos.com/calculator/6f6ckglq1w}}

\subsection{Inclusion of the vaccine in the equation}
In this section, we modify our first model such that it include a group that has a better immunity against the virus than vulnerable people, but has never had been infected. For this we need to make two version of $V$ and $I$ depending if the people are vaccinated or not. For this we index them by $v$ (vaccinated) and $u$ (unvaccinated): $V_u,V_v,I_u,I_v$. In the first section, we have use parameters $\a$ and $\b$ to quantify the affect of the virus. Now, the vaccine have an effect on that, and reduce the susceptibility to infection, let's say it reduces by a factor $\e$ the capacity of the vulnerable group to be infected by a infectious; it reduces by a factor $\d$ the capacity of the infected group to  infect a vulnerable one; it increase by a factor $\s$ the capacity of the infected group to recover and become resistant. Using our understanding of each term of the equation in section 1, we establish a new equation and detail it after for clarity :
\begin{IEEEeqnarray*}{rCl}
\dotV_u &=& -\b V_u(I_u+\e I_v) \\
\dotV_v &=& -\d\b V_v(I_u+\e I_v) \\
\dotI_u &=& \b V_u(I_u+\e I_v) -\a I_u \\
\dotI_v &=& \d\b V_v(I_u+\e I_v) -\s\a I_v. \\
\end{IEEEeqnarray*}
Here again, we use the concept of growth rate to decide the equation. The growth rate of the two vulnerable groups can be separated in two terms that represent the interactions with the vaccinated and unvaccinated infectious group. We have to take into account the fact that the vaccinated infectious group has an effect reduced by a factor $\e$ and that the vaccinated vulnerable group has a vulnerability reduced by a factor $\d$. These two terms sum up and combine the two new factors, which justify the first two lines of the system. For the two last equations, we first have to add the term of people who leave the respective vulnerable group depending if they have vaccine or not. Then we have to add the term that represent the people that leave the infected group to become resistant. It is of the same form as before, but with a factor augmentation for the vaccinated ones. The general equation for the evolution of $R$ is just the sum of the two terms we just described, it is $\dotR=-\a I_u-\s\a I_v$. As wanted, the total population $P=V_u+V_v+I_u+I_v+R$ is constant because as before, everything vanishes in $(V_u+I_u+I_v+V_v+R)'=0$.

Now we try to do the same analysis as before. We have again for the set of fixed points, the 2 dimensional subspace where $I_u=I_v=0$, and we don't have asymptotic stability on them. The constant quantity of motion doesn't come easily as before, but we see similarities. The $V$'s are decreasing, and the $I$'s may change direction during the process. We want to integrate the first two equations because we can separate at least the $V'$s from the other functions. Using $\a_v=\s\a$, $\a_u=\a$, $\b_v=\d\b$ ($\b_u=\b$), and a index $w$ that go through $\{u,v\}$,  we see that again, $(V_w+I_w)'=\a_w I_w$. We get 
\begin{IEEEeqnarray*}{rCl}
(\log V_w)' 
&=& \frac{\dotV_w}{V_w} 
= -\b_w(I_u+\e I_v)
= -\b_w(\frac{(V_u+I_u)'}{\a_u}+\e \frac{(V_v+I_v)'}{\a_v}) \\
&=& \bigg(-\b_w\Big(\frac{1}{\a_u}(V_u+I_u)+ \frac{\e}{\a_v}(V_v+I_v)\Big)\bigg)'
\end{IEEEeqnarray*}
This gives us a pair of constant quantities
\begin{equation} \label{eq:modèle-1-const}
    L_w(V_u,V_v,I_u,I_v)=\log V_w + \b_w\Big(\frac{1}{\a_u}(V_u+I_u)+ \frac{\e}{\a_v}(V_v+I_v)\Big).\end{equation}
We take a linear combination of them to make the second term disapear and obtain that $\d\log V_u - \log V_v$
is constant too, so $V_v= \frac{V_v(0)}{V_u(0)^\d}V_u^\d$. This show how the vaccine affect the vulnerable group depending on if they are vaccinated or not. Like in the first model in \prettyref{sec:premier-modèle}, $(V_w+I_w)'=\a_w I_w$ implies that $I_w$ tends to zero and $V_w$ tends to a positive value $_w(\infty)$. We take the limit of the pair of constant quantity \prettyref{eq:modèle-1-const} and obtain
$$L_0
= \log V_w(\infty) + \b_w\Big(\frac{1}{\a_u}V_u(\infty)+ \frac{\e}{\a_v}V_v(\infty)\Big)$$
We can substitute the relation $V_v= \frac{V_v(0)}{V_u(0)^\d}V_u^\d$ into it to obtain a implicit formula for these limit values. It has the same form as in the first model and a similar resolution.

In the last two sections, we have not been able to investigate the stability. Without doing an mathematical analysis, we can intuitively see that the scheme $V \to I \to R$ we choose, cannot give a equilibrium, because $V$ is monotone. People only leave the vulnerable group, but nobody enter it. A possible model to investigate that would allow to have a balance between input and output, and a possible equilibrium, is a demographic model where births and deaths are included in the modelisation as well as immigration and emigration.
%For this reason, we decide to consider a model that take into account the births and the deaths.  We consider that the "velocity" of the epidemic is slow enough to have interaction with the renewal of the population. In addition, we can consider the epidemic as located in an isolated place, and that the immigration and emigration are considered the same as birth and death. Immigration is initially vulnerable.
 
    
    \nocite{*}
    \printbibliography
\end{document}
