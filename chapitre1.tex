\chapter{premier}
The classic model of Lokta-Voltera equations is 
\[\dot{x}=x(\a-\b y) \quad  \dot{y} = y(-\g + \d x)\]
where $x$ and $y$ reprensent the size of the population of preys, and predators respectively. The parameters $\a,\b,\g,\d$ are positive scalars. We present a motivation behind this modelisation by showing how we can arrive to this choice of equation:

\section{Motivation} \label{sec:motiv}
We remind that for a function $x$ of one variable $t$, 
\[x'(t) = \frac{\dd}{\dd t}x(t) = \dotx(t) \lim_{h\to0}\frac{x(t+h)-x(t)}{h}\] 
is the derivative when the limit exist. Qualitatively it indicate the amount of change of the function with respect to the variable. This amount is absolute and doesn't depend of the size of x. In consequence, when $x$ is non null which is an assumption we will always make for initial value, $\frac{\dot{x}}{x}$ quantify the relative rate of change of the function, given by the mean number of descendant of a individual. For example if a specie reproduce itself always with the same speed no matter the population or the environment, we can say that their grow rate is a constant c:
\[ \dot{x} = cx, \]
If the population is positive, 
\[ \frac{\dot{x}}{x} = \frac{d}{dt}(\log{x}) = c \]
and then by integrating,
\[ \log{x(t)} = ct + \log{x(0)}, \]
giving us $x(t) = x(0)e^{ct}$, the exponential growth.

Obviously this is doesn't encapsulate the reality as the function grow very fast forever. The growth rate must decrease as the population increase. This come from multiple complex reasons such as envirenment capacities in food, space etc\dots For now we suppose it is from the simplest form, a linear decrease of this growth rate: 
\[ \dot{x} = x(\a -\b x) \]
That give us the logistic growth, where $\a$ represent the initial growth rate, and $\b$ change how fast the growth rate slow down as the population size increase. Here we have one non trivial equilibrium when $\dot{x(t)}=0$ i.e. when $x(t)=\a/\b=x_*$. If not, we remark that $\dotx > 0$ when $ x < x_*$, and  $\dotx < 0$ when $ x < x_*$, meaning that $x_*$ seems stable. indeed we do the computations :
\begin{IEEEeqnarray*}{rCl}
   \frac1\a = \frac{\dotx}{x(1-x/x_*)}
   = \frac{\dotx}{x} + \frac{\dotx}{(x_* - x)}
   = (\log|x| + \log|x_* - x|)'.
\end{IEEEeqnarray*}
By integrating from 0 to t we get
\[ t/\a  = \log(x(t)) - \log|x_* -x(t)| -( \log(x_0) - \log|x_* -x_0|)
= \log \frac{x(t)(x_* -x_0)}{x_0(x_*-x(t))} \]
and rearranging terms
\[x(t) = \frac{x_0x_*e^{\a t}}{x_* + x_0(e^{\a t}-1)} \]
which is well defined for $t\in\R$ if $0<x_0<x_*$, and for $t\in[1/\a \log(1 - x_*/x_0),\infty]$ if $0<x_*<x_0$. We see that in both cases $x(t)\to x_*$ as $t\to\infty$.

In conclusion for this logistic growth, the population will always stabilise in the direction of a unique non trivial equilibrium.


Now we want to introduce a second specie, the predator which alter the growth of the prey population. That mean that the growth rate will decrease in function of the prey, let's say linearly :
\[ \dotx = x(\a -\b y).\]
In the opposite, the growth rate of the predator increase together with the population. But it decrease without them :
\[ \doty = x(-\g +\d x).\]
This is the Lokta-Voltera equations for a prey-predator system.

\section{*Le papier sur les phages marins*}
The idea is that we want to be able to modify the way the size of the species affect the rate. Traditional Lokta-Voltera equation consider the rate of growth as linear in the species. We use then functions $F$ and $G$ that are strictly monotonic,hence bijective, and increase from $0$ to $\infty$ to replace the linearity:
\begin{equation} \label{eq:LV}
    \begin{cases}
    \dotB = F(B)(\a -\b G(\P)) \\
    \dotP = G(\P)(-\g +\d F(B))
    \end{cases}
\end{equation}
Here again, $\a,\b,\g,\d$ are positive scalars.
\\ \\
We search for non trivial equilibrium where $\dotB=\dotP=0$ :
\[ (B_* = (F^{-1}(\a/\b) \quad \P_* = G^{-1}(\g/\d) ) \]
This let us rewrite \prettyref{eq:LV} using this notation :
\begin{equation} \label{eq:LV*}
    \begin{cases}
    \dotB = \b F(B)(F(B_*) - G(\P)) \\
    \dotP = \d G(\P)(-G(\P_*) + F(B))
    \end{cases}
\end{equation}
Now we can study the sign of the derivatives $\dotB$ and $\dotP$ and draw a phase plane. The positive values $B_*$ and $\P_*$ divide the positive plane $\R_+^2$ in four regions.
[phase plane]
Trajectories seem to turn around the center of equilibrium and we search for a first integral by taking the cross product of \prettyref{eq:LV*} and dividing by $F(B)G(\P)$ :
\begin{IEEEeqnarray*}{rCl} 
    0 &=& \big(\dotB \d G(\P)(-G(\P_*) + F(B))
        - \dotP \b F(B)(F(B_*) - G(\P)) \big) \frac{1}{F(B)G(\P)} 
        \IEEEyesnumber \label{eq:1integral} \\
    &=& \d(-\frac{\dotB}{F(B)}G(\P_*) + \dotB)
    -  \b(\frac{\dotP}{G(\P)}F(B_*) - \dotP) \\
    &=& \Big(\d (-P(B)G(\P_*) + B) -\b(Q(\P) F(B_*))-\P)\Big)'
\end{IEEEeqnarray*}
where $P$ and $Q$ are primitives of $1/F$ and $1/G$, and exist because $F$ and $G$ are continuous. This give us a conserved quantity 
\begin{IEEEeqnarray*}{rCl}
V(B,\P) &=& \d \big(-P(B)G(\P_*) + B\big) 
-\b\big(Q(\P) F(B_*)-\P\big) \\
&=& V(B(0),\P(0)) \\
&=:& V_0
\end{IEEEeqnarray*}
Now we know then that $(B,\P) \in V^{-1}(\{V_0\})$, a closed set as V is continuous. [arguments for the fact that this is a cyclic orbit under asumptions on F and G][plots].
Now that we know that it's cyclic with a period $\tau$, we derive a modified Voltera principle by integrating these qualities obtained from \prettyref{eq:LV*} :
\[ \dotB / F(B) = \b (F(B_*) - G(\P)). \]
the left side gives
\begin{IEEEeqnarray*}{rCl}
    \int_0^\tau \frac{\dotB}{F(B)} 
     &=& \int_0^\tau (P(B))'
     = P(B(\tau))-P(B(0)) =  P(B(0))-P(B(0)) =0,
\end{IEEEeqnarray*}
and then the rigth side
  \[ 0= \int_0^\tau \b (F(B_*) - G(\P))
    = \tau\b F(B_*) - \int_0^\tau \b  G(\P) \]
which implies that
 \[  F(B_*) = \frac1\tau\int_0^\tau G(\P). \]
 Note that if $G=$Id, this says that the mean of the population $\P$ during time is $F(B_*)$. By the same argument with the second equation, we obtain that similarly
  \[  G(\P_*) = \frac1\tau\int_0^\tau F(B). \]
  \\ \\
  In \prettyref{sec:motiv}, we explained how the own size of a population can affect its growth. We presented the logistic growth consequence of this equation :
  \[\dotx = x(\a - \b x)\]
  We add then a term in the rates of \prettyref{eq:LV} to modelise the fact that the growth rate of a population decrease with the size of population due to environment capabilities or competition. Again we use $F$ and $G$ to quantify their importance :
  \begin{equation} \label{eq:LV3}
    \begin{cases}
    \dotB &= F(B)(\a -\b G(\P) - \m F(B)) \\
    \dotP &= G(\P)(-\g +\d F(B) - \n G(\P))
    \end{cases}
\end{equation}
with new positive scalars $\m$ and $\n$. We search for non trivial equilibrium where $\dotB_{**}=\dotP_{**}=0$ and obtain a linear system in $F(B_{**})$ and $G(\P_{**})$ :
\begin{equation*}
    \begin{cases}
    -\a = - \m F(B_{**}) -\b G(\P_{**}) \\
    \g = \d F(B_{**}) - \n G(\P_{**}) 
    \end{cases}
\end{equation*}
which give
\begin{IEEEeqnarray*}{rCl}
    \m\g-\d\a &=& \m\d F(B_{**}) - \m\n G(\P_{**}) - \d\m F(B_{**}) -\d\b G(\P_{**}) \\  &=& (-\m\n-\d\b) G(\P_{**}) 
\end{IEEEeqnarray*}
and then as $\m\n+\d\b > 0$ and supposing $\a\d>\g\m$
\[ G(\P_{**}) = \frac{\a\d-\g\m}{\b\d+\n\m} ,\quad
\P_{**}=P^{-1}\Big(\frac{\a\d-\g\m}{\b\d+\n\m}\Big) \]
Similarly,
\[ F(B_{**}) = \frac{\b\g+\n\a}{\b\d+\n\m}, \quad
B_{**} = F^{-1} \Big(\frac{\b\g+\n\a}{\b\d+\n\m}\Big) \]
Here we cannot derive a first integral like we did in \prettyref{eq:1integral} by separating variables $B$ and $\P$. Instead, we want to test the stability of $(\Bstar,\Pstar)$. For this we develop the theory of stability of Lianupov.
\\ \\
Consider a differential equation $\bdotx=\mathbf{F}(\mathbf{x})$, such that there exist a unique solution for each initial point and for all $t\geq0$. Such solutions are denoted by the flow $\phi$, such that $t\mapsto \phi(\mathbf{x}_0,t)$ is the solution initialised at $\mathbf{x}_0$. 

\begin{definition}
 A fixed point $\mathbf{x}_*$ of  is said \emph{Lyanupov stable} or \emph{L-stable} if 
 for all $\epsilon>0$, there exists a $\delta>0$ s.t. for all $\mathbf{x_0}$ and for all $t>0$, $\|\mathbf{x_0} - \mathbf{x}_*\| < \delta$ implies $\|\phi(\mathbf{x_0},t) - \mathbf{x}_*\| < \epsilon$.
\end{definition}

\begin{definition}
A fixed point $\mathbf{x}_*$ of  is said \emph{attracting} if there exists a $\delta>0$ s.t. for all $\mathbf{x_0}$, $\|\mathbf{x_0} - \mathbf{x}_*\| < \delta$ implies that $\phi(\mathbf{x_0},t) \to \mathbf{x}_*$ as $t \to\infty$.
\end{definition}

\begin{definition}
    A fixed point which is L-stable and attracting is said \emph{asymptoticaly stable}
\end{definition}

\begin{remarque}
Note that these two notions are different. For an example of non-attracting point which is L-stable, we can take simply $F(x)=0$ or for a non trivial case, we can take $\dotx = -y$, $\dot{y} = x$ who describe the circle trajectories $x(t)=cos(t)$, $y(t)=sin(t)$.

In the other way there exist non L-stable points which are attracting. Such a point is the limit of all near trajectories but they always go far before converging, like a detour. For this we place our-self in polar coordinates. We want the trajectories to follow the circle and finish in $(1,0)$ for this we make the $\theta$ always go and stop to $2\pi$, and $r$ go and stop to 1. For this we can write $(\dot{r},\dot{\theta})=(1-r,2\pi-\theta) = G(r,\theta)$. But if we want $(G_1\cos(G_2),G_1\sin(G_2))$ to be continuous on $\R_+\times 0$, we should write $\dot{r} = r(1-r)$, $\dot{\theta} = \theta(2\pi-\theta)$. And to obtain the continuity of the derivative, in purpose to have a flow, we should write $\dot{r} = r(1-r^2)$, $\dot{\theta} = \theta(2\pi-\theta^2)$. This gives us what we need but to be able to explicitly change the coordinates into cartesian, we prefer $\dot{y} = \sin(\theta/2)^2$.
\end{remarque}

Now we can introduce a tool that will be useful to understand the limit comportment of the trajectories and will a tool to proove asymptotic stability.

\begin{definition}
    Assume $\mathbf{x}_*$ is a fixed point of a equation $\bdotx = \mathbf{F}(\mathbf{x})$. A function $f:U\to\R$ defined in a neighbourhood of $\mathbf{x}_*$, is said Lyanupov if \[L(\mathbf{x})>L(\mathbf{x}_*)\] 
    for all $\mathbf{x}$ in $U$, and 
    \[ \dot{L}(\phi(\mathbf{x},0)) <0= \dot{L}(\phi(\mathbf{x}_*,0))\]
    for all $\mathbf{x}$ in $U\backslash\{\mathbf{x}_*\}$
\end{definition}

\begin{theoreme}
Let $\mathbf{x}_*$ a fixed point of the differential equation $\bdotx = \mathbf{F}(\mathbf{x})$ and $F$ a Lyanupov function on $U\ni\mathbf{x}_*$. Then $\mathbf{x}_*$ is asymptotically stable.
\end{theoreme}
\begin{proof}
I'm god.
\end{proof}